\documentclass[12pt,a4paper]{article}

\input{../../../lib/doc/pkg/common}
\input{../../../lib/doc/math/operators}

\geometry{left=3cm,right=2cm,top=2cm,bottom=2cm}

\titlelabel{\thetitle. }
\patchcmd{\appendices}{\quad}{. }{}{}

\hypersetup{
	colorlinks,
	citecolor=black,
	filecolor=black,
	linkcolor=black,
	urlcolor=black
}

\numberwithin{equation}{section}

\begin{document}

    \newcommand\blanktextfield[2]{$\underset{\text{#1}}{\text{\underline{\hspace{#2}}}}$}

\makeatletter
\begin{titlepage}

	\large\newpage

    \noindent\centering{
    	МИНИСТЕРСТВО ОБРАЗОВАНИЯ И НАУКИ РОССИЙСКОЙ ФЕДЕРАЦИИ

    	Федеральное государственное автономное образовательное учреждение высшего образования \enquote{Национальный исследовательский Нижегородский государственный университет им. Н.И. Лобачевского}
    }

	\vspace*{50pt}

	Физический факультет \\[\baselineskip]

	Кафедра информационных технологий\\
	в физических исследованиях

	\vspace*{100pt}

	{\Large\textbf{Тепловое излучение в сферическом резонаторе}}

	\vspace*{\fill}

	\hfill\begin{minipage}{20em}
    	Выпускная квалификационная работа\\
		студента 4 курса бакалавриата 05144 группы\\
		\textbf{Василевского А.В.}
    \end{minipage} \\[\baselineskip]

	\hfill\begin{minipage}{20em}
		Основная профессиональная образовательная
		программа подготовки бакалавров по
		направлению 09.03.02~--- \enquote{Информационные системы и технологии}
		(профиль программы: \enquote{Информационные системы и технологии в физических исследованиях})
    \end{minipage}

	\vspace*{\fill}

	\hfill\begin{minipage}{15em}
		\blanktextfield{(подпись)}{1in} Василевский А.В.\\[\baselineskip]
		Научный руководитель:\\
		доцент кафедры ИТФИ\\
		к. ф.-м. н.\\[\baselineskip]
		\blanktextfield{(подпись)}{1in} Бурланков Д.Е.
    \end{minipage}

	\vspace*{\fill}

	Нижний Новгород\\
	2018

\end{titlepage}
\makeatother


    \tableofcontents

    \newpage

    \includedoc{.}{../../fragments/contents-intro}

    \section{Постановка задачи}
    \includedoc{../../fragments/spherical_resonators}{contents}

    \section{Математическое введение}
    \includedoc{../../fragments/math}{contents}

    \section{Нахождение сферических мод}
    \includedoc{../../fragments/spherical_modes}{contents}

    \section{Термодинамика сферических резонаторов}
    \includedoc{../../fragments/thermodynamics}{contents}

    \includedoc{.}{../../fragments/contents-outro}

    \clearpage

    \begin{appendices}
        \includedoc{../../fragments/thermodynamics}{contents-appendix}
    \end{appendices}

    \clearpage

    \phantomsection
    \addcontentsline{toc}{section}{Список литературы}
    \bibliographystyle{../../../lib/doc/bib/utf8gosttu}
    \bibliography{../../../lib/doc/bib/math,../../../lib/doc/bib/resonators,../../../lib/doc/bib/physics}

\end{document}
