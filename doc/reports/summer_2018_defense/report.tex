\documentclass[12pt,a4paper]{article}

\usepackage[T1,T2A]{fontenc}
\usepackage[utf8]{inputenc}
\usepackage[english,russian]{babel}
\usepackage{microtype}
\usepackage{csquotes}
\usepackage{amsmath}
\usepackage{amsthm}
\usepackage{amssymb}
\usepackage{mathtext}
\usepackage{physics}
\usepackage{newfloat}
\usepackage{caption}
\usepackage{indentfirst}
\usepackage{titlesec,titletoc}
\usepackage{geometry}
\usepackage{hyperref}
\usepackage{mdframed}
\usepackage[inline]{enumitem}
\usepackage{graphicx}
\usepackage{subfig}
\usepackage[titletoc,toc]{appendix}

\DeclareGraphicsExtensions{.pdf,.png,.jpg,.PNG}
\graphicspath{{./img/}}
\captionsetup[figure]{justification=centering}
\renewcommand{\thesubfigure}{\asbuk{subfigure}}
\DeclareCaptionLabelSeparator{dotseparator}{. }
\captionsetup{labelsep=dotseparator}
\geometry{left=1cm,right=1cm,top=2cm,bottom=2cm}
\makeatletter\appto{\appendices}{\def\Hy@chapapp{Appendix}}\makeatother
\renewcommand{\appendixtocname}{Приложения}
\renewcommand{\appendixpagename}{Приложения}

\DeclareMathOperator{\Rot}{\mathbf{rot}}
\DeclareMathOperator{\Grad}{\mathbf{grad}}
\DeclareMathOperator{\Div}{\mathbf{div}}
\DeclareMathOperator{\D}{D}
\newcommand{\V}[1]{\mathbf{#1}}
\newcommand{\Op}[1]{\hat{\V{#1}}}


\geometry{left=3cm,right=2cm,top=2cm,bottom=2cm}

\titlelabel{\thetitle. }
\patchcmd{\appendices}{\quad}{. }{}{}

\hypersetup{
	colorlinks,
	citecolor=black,
	filecolor=black,
	linkcolor=black,
	urlcolor=black
}

\numberwithin{equation}{section}

\begin{document}

    \makeatletter
\begin{titlepage}

	\newpage

    \noindent\centering{
    	МИНИСТЕРСТВО ОБРАЗОВАНИЯ И НАУКИ РОССИЙСКОЙ ФЕДЕРАЦИИ

    	Федеральное государственное автономное образовательное учреждение высшего образования \enquote{Нижегородский государственный университет им. Н.И. Лобачевского} (ННГУ)
    }

	\vspace*{50pt}

	Физический факультет \\[\baselineskip]

	Кафедра информационных технологий\\
	в физических исследованиях

	\vspace*{100pt}

	{\Large\textbf{Отчет}} \\
	по преддипломной практике \\[\baselineskip]

	{\Large\textbf{Тепловое излучение в сферическом резонаторе}}

	\vspace*{\fill}

	\hfill\begin{minipage}{15em}
    	Выполнил\\
		студент 4 курса 05144 группы\\
		\textbf{Василевский А.В.}
    \end{minipage} \\[\baselineskip]

	\hfill\begin{minipage}{15em}
    	Научный руководитель\\
		доцент кафедры ИТФИ\\
		к. ф.-м. н.\\
		\textbf{Бурланков Д.Е.}
    \end{minipage}

	\vspace*{\fill}

	Нижний Новгород\par

	\centering{2018 г.}

\end{titlepage}
\makeatother


    \tableofcontents

    \newpage

    \includedoc{.}{../../fragments/contents-intro}

    \section{Постановка задачи}
    \includedoc{../../fragments/spherical_resonators}{contents}

    \section{Математическое введение}
    \includedoc{../../fragments/math}{contents}

    \section{Нахождение сферических мод}
    \includedoc{../../fragments/spherical_modes}{contents}

    \section{Термодинамика сферических резонаторов}
    \includedoc{../../fragments/thermodynamics}{contents}

    \includedoc{.}{../../fragments/contents-outro}

    \clearpage

    \begin{appendices}
        \includedoc{../../fragments/thermodynamics}{contents-appendix}
    \end{appendices}

    \clearpage

    \nocite{*}
    \bibliographystyle{../../../lib/doc/bib/utf8gost705s}
    \bibliography{../../../lib/doc/bib/math,../../../lib/doc/bib/resonators,../../../lib/doc/bib/physics}

\end{document}
