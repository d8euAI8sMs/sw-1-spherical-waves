\documentclass[12pt,a4paper]{article}

\usepackage[T1,T2A]{fontenc}
\usepackage[utf8]{inputenc}
\usepackage[english,russian]{babel}
\usepackage{microtype}
\usepackage{csquotes}
\usepackage{amsmath}
\usepackage{amsthm}
\usepackage{amssymb}
\usepackage{mathtext}
\usepackage{physics}
\usepackage{newfloat}
\usepackage{caption}
\usepackage{indentfirst}
\usepackage{titlesec,titletoc}
\usepackage{geometry}
\usepackage{hyperref}
\usepackage{mdframed}
\usepackage[inline]{enumitem}
\usepackage{graphicx}
\usepackage{subfig}
\usepackage[titletoc,toc]{appendix}

\DeclareGraphicsExtensions{.pdf,.png,.jpg,.PNG}
\graphicspath{{./img/}}
\captionsetup[figure]{justification=centering}
\renewcommand{\thesubfigure}{\asbuk{subfigure}}
\DeclareCaptionLabelSeparator{dotseparator}{. }
\captionsetup{labelsep=dotseparator}
\geometry{left=1cm,right=1cm,top=2cm,bottom=2cm}
\makeatletter\appto{\appendices}{\def\Hy@chapapp{Appendix}}\makeatother
\renewcommand{\appendixtocname}{Приложения}
\renewcommand{\appendixpagename}{Приложения}

\DeclareMathOperator{\Rot}{\mathbf{rot}}
\DeclareMathOperator{\Grad}{\mathbf{grad}}
\DeclareMathOperator{\Div}{\mathbf{div}}
\DeclareMathOperator{\D}{D}
\newcommand{\V}[1]{\mathbf{#1}}
\newcommand{\Op}[1]{\hat{\V{#1}}}


\title{Математическое приложение}
\author{Василевский А.В.}

\begin{document}

    \maketitle
    \tableofcontents

    \appendix

    %
    %
    %
    %%%%%%%%%%%%%%%%%%%%%%%%%%%%%%%%%%%%%%%%%%%%%%%%%%%%%%%%%%%%%%%%%%%%%%%
    %                           SECTION                                   %
    %%%%%%%%%%%%%%%%%%%%%%%%%%%%%%%%%%%%%%%%%%%%%%%%%%%%%%%%%%%%%%%%%%%%%%%
    %
    %
    %

    \section{Волновое уравнение}

        Получим стационарное волновое уравнение в свободном пространстве для монохроматической волны в нерелятивистском приближении.

        Исходная система уравнений Максвелла имеет вид:
        %
        \begin{equation}\begin{aligned}\label{eq:maxwell_empty_space}
            \Rot\V{E} &= - \frac{1}{c} \frac{\partial \V{B}}{\partial t}, \\
            \Rot\V{B} &= \frac{1}{c} \frac{\partial \V{E}}{\partial t}, \\
            \Div\V{B} &= 0, \\
            \Div\V{E} &= 0.
        \end{aligned}\end{equation}
        %
        При
        %
        \begin{equation}\begin{aligned}
            \V{E}(\V{r}, t) &= \V{\hat{E}}(\V{r}) \exp(- i \omega t), \\
            \V{B}(\V{r}, t) &= \V{\hat{B}}(\V{r}) \exp(- i \omega t)
        \end{aligned}\end{equation}
        %
        уравнения принимают вид
        %
        \begin{equation}\begin{aligned}
            \Rot\V{\hat{E}} &= i \omega \frac{1}{c} \V{\hat{B}}, \\
            \Rot\V{\hat{B}} &= - i \omega \frac{1}{c} \V{\hat{E}}.
        \end{aligned}\end{equation}
        %
        Заметим, что вторые два уравнения удовлетворяются в автоматически. Взятием ротора от первого уравнения и подстановкой в него второго уравнения системы можно получить следующее выражение для вектора $\V{E}$ (аналогичное получается и для $\V{H}$):
        %
        \begin{equation}
            \Rot\Rot{\V{\hat{E}}} = i \omega \frac{1}{c} \left( -i \omega \frac{1}{c} \V{\hat{E}} \right) = \frac{\omega^2}{c^2} \V{\hat{E}}.
        \end{equation}
        %
        Полученное уравнение называется стационарным волновым уравнением.

        Его можно упростить, принимая во внимание вторую пару уравнений \autoref{eq:maxwell_empty_space} и тождество
        %
        \begin{equation}\label{eq:laplacian_vect}
            \Rot\Rot = \Grad\Div - \Delta,
        \end{equation}
        %
        где $\Delta$ -- векторный оператор Лапласа:
        %
        \begin{equation}
            \Delta\V{\hat{E}} = - \frac{\omega^2}{c^2} \V{\hat{E}}.
        \end{equation}

        В дальнейшем условимся опускать знак \enquote{$\hat{\ }$}, имея в виду именно не зависящую от времени часть $\V{E}$, если не сказано обратного.

    %
    %
    %
    %%%%%%%%%%%%%%%%%%%%%%%%%%%%%%%%%%%%%%%%%%%%%%%%%%%%%%%%%%%%%%%%%%%%%%%
    %                           SECTION                                   %
    %%%%%%%%%%%%%%%%%%%%%%%%%%%%%%%%%%%%%%%%%%%%%%%%%%%%%%%%%%%%%%%%%%%%%%%
    %
    %
    %

    \section{Векторный оператор Лапласа в произвольных координатах}

        Ограничимся трехмерным римановым пространством. Получим явный инвариантный вид введенного оператора Лапласа. Будем опираться на формулировку \autoref{eq:laplacian_vect}.

        Цель данного пункта -- показать, что введенный выше векторный оператор Лапласа не является особым объектом, принадлежит к классу дифференциальных операторов второго порядка, а его вид в координатной записи не зависит от поля, к которому он применяется.

        Ротор в криволинейных координатах через ковариантные производные от ковариантных компонент вектора $\V{a}$ представлен в виде
        %
        \begin{equation}
            \left( \Rot\V{a} \right)^i
                = \frac{1}{2} \varepsilon^{ijk} \left(
                    \nabla_k a_j - \nabla_j a_k
                \right)
                \equiv \varepsilon^{ijk} \nabla_{[k} a_{j]}.
        \end{equation}
        %
        Распишем $\Rot\Rot\V{a}$:
        %
        \begin{equation}
            \left( \Rot\Rot\V{a} \right)^i
                = \varepsilon^{ijk} \nabla_{[k} \left( \Rot\V{a} \right)_{j]}.
        \end{equation}
        %
        Получим явный вид альтернированной ковариантной производной ротора:
        %
        \begin{equation}\begin{aligned}
            \nabla_{[k} \left( \Rot\V{a} \right)_{j]}
                &= \nabla_{[k} \left( g_{j]m} \left( \Rot\V{a} \right)^m \right) \\
                &= g_{[jm} \nabla_{k]} \left( \Rot\V{a} \right)^m \\
                &= g_{[jm} \nabla_{k]} \left(
                       \varepsilon^{mqr} \nabla_{[r} a_{q]}
                \right) \\
                &= g_{[jm} \varepsilon^{mqr} \left(
                   \nabla_{k]} \nabla_{[r} a_{q]}
                \right),
        \end{aligned}\end{equation}
        %
        откуда
        %
        \begin{equation}
            \left( \Rot\Rot\V{a} \right)^i
                = \varepsilon^{ijk} g_{[jm} \varepsilon^{mqr} \left(
                    \nabla_{k]} \nabla_{[r} a_{q]}
                \right).
        \end{equation}
        %
        Распишем произведение перед скобкой, пользуясь тождеством
        %
        \begin{equation}
           \varepsilon_{ijk} \varepsilon^{imn} = \delta_j^m \delta_k^n - \delta_j^n \delta_k^m,
        \end{equation}
        %
        получим
        %
        \begin{equation}\begin{aligned}
           \varepsilon^{ijk} g_{jm} \varepsilon^{mqr}
                &= \varepsilon^{kij} g_{jm} \varepsilon^{mqr} \\
                &= g^{ip} g^{ks} \varepsilon_{msp} \varepsilon^{mqr} \\
                &= g^{ip} g^{ks} \left(
                    \delta_s^q \delta_p^r - \delta_s^r \delta_p^q
                \right)
        \end{aligned}\end{equation}
        %
        и аналогично для альтернированного по индексам $[jk]$ варианта
        %
        \begin{equation}\begin{aligned}
           \varepsilon^{ijk} g_{km} \varepsilon^{mqr}
                &= \varepsilon^{ijk} g_{jm} \varepsilon^{mqr} \\
                &= g^{ip} g^{js} \varepsilon_{mps} \varepsilon^{mqr} \\
                &= g^{ip} g^{js} \left(
                    \delta_p^q \delta_s^r - \delta_p^r \delta_s^q
                \right) \\
                &= - g^{ip} g^{js} \left(
                    \delta_s^q \delta_p^r - \delta_s^r \delta_p^q
                \right).
        \end{aligned}\end{equation}

        В итоге, в явном виде ротор ротора перепишется следующим образом:
        %
        \begin{equation}\begin{aligned}
            \left( \Rot\Rot\V{a} \right)^i
                &= \frac{1}{2} g^{ip} \left\{
                       g^{ks} \left(
                           \delta_s^q \delta_p^r - \delta_s^r \delta_p^q
                       \right) \nabla_k \nabla_{[r} a_{q]} +
                       g^{js} \left(
                           \delta_s^q \delta_p^r - \delta_s^r \delta_p^q
                       \right) \nabla_j \nabla_{[r} a_{q]}
                   \right\} \\
                &= g^{ip} \left(
                       g^{kq} \nabla_k \nabla_{[p} a_{q]} - g^{kr} \nabla_k \nabla_{[r} a_{p]}
                   \right) \\
                &= 2 g^{ip} g^{kq} \left(
                       \nabla_k \nabla_{[p} a_{q]}
                   \right) \\
                &= g^{ip} g^{kq} \left(
                       \nabla_k \nabla_{p} a_{q} - \nabla_k \nabla_{q} a_{p}
                   \right) \\
                &= g^{ip} \nabla_k \nabla_{p} a^k - g^{kq} \nabla_k \nabla_{q} a^i
        \end{aligned}\end{equation}
        %
        Пользуясь теперь
        %
        \begin{equation}\begin{gathered}
            (\nabla_k \nabla_p - \nabla_p \nabla_k) a^i = - R_{kpm}^{\hphantom{kpm}i} a^m , \\
            R_{kpm}^{\hphantom{kpm}i} \delta_i^k = R_{kpm}^{\hphantom{kpm}k} = R_{pm} ,
        \end{gathered}\end{equation}
        %
        где $R_{kpm}^{\hphantom{kpm}i}$ -- тензор кривизны (тензор Римана-Кристоффеля), а $R_{pm}$, соответственно, тензор Риччи, преобразуем последнюю строчку к виду
        %
        \begin{equation}\begin{aligned}
            \left( \Rot\Rot\V{a} \right)^i
                &= g^{ip} \nabla_k \nabla_p a^k - g^{kq} \nabla_k \nabla_q a^i \\
                &= g^{ip} \nabla_p \nabla_k a^k
                    + g^{ip} R_{pm} a^m
                    - g^{kq} \nabla_k \nabla_q a^i \\
                &= \left( \Grad\Div{\V{a}} - \Delta{\V{a}} \right)^i + g^{ip} R_{pm} a^m
        \end{aligned}\end{equation}
        %
        В пространствах с нулевой кривизной (а мы рассматриваем такие пространства) тензор Риччи тождественно равен нулю.

        Итак, мы получили общий вид оператора Лапласа в произвольной системе координат. Выпишем его отдельно:
        %
        \begin{equation}
            \Delta{\V{a}} = g^{kq} \nabla_k \nabla_{q} a^i
                          = \nabla^k \nabla_k a^i
        \end{equation}
        %
        Мы также видим, что вид оператора Лапласа не зависит от объекта, к которому он применяется (скалярное, векторное, тензорное поле). Оператор Лапласа является частным случаем более общего семейства линейных дифференциальных операторов порядка $p$, порождаемых определенным тензорным полем $a$:
        %
        \begin{equation}
            \D_a\tau = a^{i_1 \dots i_p} \nabla_{i_1} \dots \nabla_{i_p} \tau ,
        \end{equation}
        %
        где под $\tau$ понимается тензорное поле произвольной (в т.ч. нулевой) валентности. Ниже будут рассмотрены коммутационные соотношения и условия коммутации между операторами вплоть до операторов второго порядка.

    %
    %
    %
    %%%%%%%%%%%%%%%%%%%%%%%%%%%%%%%%%%%%%%%%%%%%%%%%%%%%%%%%%%%%%%%%%%%%%%%
    %                           SECTION                                   %
    %%%%%%%%%%%%%%%%%%%%%%%%%%%%%%%%%%%%%%%%%%%%%%%%%%%%%%%%%%%%%%%%%%%%%%%
    %
    %
    %

    \section{Ли-вариация, поля Киллинга}

        Будем рассматривать трехмерное пространство без кручения\footnotemark{}.

        \footnotetext{
            Речь идет о римановом пространстве, в котором все связности естественным образом симметричны, т.е. $\Gamma_{cp}^a = \Gamma_{pc}^a$. См. например, \cite{riemannian_geometry_and_tensor_analysis}.
        }

        Ли-вариация является инвариантным обобщением производной по направлению на произвольные тензорные поля \cite{lie_derivative_theory,symmetry_and_killing_fields}. Для произвольного тензорного поля $\tau(P)$ вводится оператор
        %
        \begin{equation}
            \delta_\xi \tau(P) = \lim\limits_{\varepsilon \to 0} \frac{
                \tau(P + \varepsilon \xi) - \tau(P)
            }{\varepsilon}.
        \end{equation}
        %

        Второе эквивалентное определение (\cite{lie_derivative_theory}) состоит в следующем. Рассмотрим тензорное поле $\tau(\xi)$ и бесконечно малое координатное преобразование $T: \xi \rightarrow {'\xi}$ такое, что в заданной некоторой системе $\varkappa$ $T$ выражается преобразованием $'\xi^i = \xi^i + v^i d\varepsilon$, где $v^i$ -- некоторое векторное поле. При применении преобразования координат $T^{-1}: \varkappa \rightarrow {'\varkappa}$, причем $'\xi^i \rightarrow \xi^{i'}$, мы получим тензор $'\tau^{'\varkappa}(\xi)$, заданный, однако, в $'\varkappa$. Наконец, рассмотрим тензор $'\tau(\xi)$, заданный в $\varkappa$ и имеющий те же координаты, что и полученный $'\tau^{'\varkappa}(\xi)$ в $'\varkappa$. Ли-вариацией тензорного поля $\tau(\xi)$ по векторному полю $v$ называется разность между $\tau(\xi)$ и $'\tau(\xi)$, деленная на $d\varepsilon$.

        Ли-вариация тензорного поля ${T^{a_1 \dots a_p}}_{b_1 \dots b_q}(M)$ по направлению векторного поля $\xi(M)$ в координатной записи имеет вид:
        %
        \begin{equation}\begin{aligned}
            \delta_\xi {T^{a_1 \dots a_p}}_{b_1 \dots b_q}
                &= \xi^c \left( \partial_c {T^{a_i \dots a_p}}_{b_1 \dots b_q} \right) \\
                &- \left( \partial_{c} \xi^{a_1} \right) {T^{c a_2 \dots a_p}}_{b_1 \dots b_q} - \dots
                 - \left( \partial_{c} \xi^{a_p} \right) {T^{a_1 \dots a_{p-1} c}}_{b_1 \dots b_q} \\
                &+ \left( \partial_{b_1} \xi^c \right) {T^{a_1 \dots a_p}}_{c b_2 \dots b_q} + \dots
                 + \left( \partial_{b_q} \xi^c \right) {T^{a_1 \dots a_p}}_{b_1 \dots b_{q-1} c} ,
        \end{aligned}\end{equation}
        %
        где введено $\partial_k = \pdv*{x^k}$. В частности, для векторных полей справедливо
        %
        \begin{equation}\begin{aligned}\label{eq:vector_field_commutator}
            \delta_\xi T^a
                = \xi^c \partial_c T^a - T^c \partial_{c} \xi^a
                \equiv [\xi, T]^a.
        \end{aligned}\end{equation}
        %
        Этим вводится понятие коммутатора двух векторных полей (Ли-коммутатор, скобка Ли).

        В приведенных выше выражениях в пространствах с линейной связностью частная производная $\partial_k$ может быть заменена на абсолютную (ковариантную) $\nabla_k$. Действительно, для один раз контравариантного тензора
        %
        \begin{equation}\begin{aligned}
            \qty[\xi, \eta]^a
                &= \xi^c \nabla_c \eta^a - \eta^c \nabla_{c} \xi^a \\
                &= \xi^c \qty( \pdv{\eta^a}{x^c} + \Gamma_{cp}^a \eta^p )
                    - \eta^c \qty( \pdv{\xi^a}{x^c} + \Gamma_{cp}^a \xi^p ) \\
                &= \xi^c \pdv{\eta^a}{x^c} - \eta^c \pdv{\xi^a}{x^c}
                    + \qty( \xi^c \eta^p \Gamma_{cp}^a - \eta^c \xi^p \Gamma_{cp}^a ) .
        \end{aligned}\end{equation}
        %
        Последняя скобка обращается в нуль после замены во втором слагаемом $c \leftrightarrow p$ ввиду симметрии связности по нижним индексам: $\Gamma_{cp}^a = \Gamma_{pc}^a$.

        Векторные поля, на которых Ли-вариация метрического тензора равна нулю, называются движениями, или полями Киллинга:
        %
        \begin{equation}\begin{aligned}
            \delta_\xi g_{ab}
                &= \xi^c \nabla_c g_{ab} + g_{cb} \nabla_{c} \xi^a + g_{ac} \nabla_{c} \xi^a \\
                &= g_{cb} \nabla_{a} \xi^c + g_{ac} \nabla_{b} \xi^c \\
                &= \nabla_{a} \xi_b + \nabla_{b} \xi_a \\
                &= 2 \nabla_{(a} \xi_{b)}.
        \end{aligned}\end{equation}

        Ли-вариация $\delta_\xi$ на поле Киллинга $\xi$ называется оператором Киллинга и обозначается для краткости $\Op{\xi}$.

        Найдем векторы Киллинга трехмерного риманова пространства \cite{symmetry_and_killing_fields}. Запишем условие инвариантности метрики:
        %
        \begin{equation}
            \delta_\xi g_{ab}
                = 2 \nabla_{(a} \xi_{b)}
                = 0
                \Leftrightarrow \nabla_{(a} \xi_{b)} = 0.
        \end{equation}
        %
        Продифференцируем полученное выражение
        %
        \begin{equation}
            \nabla_c \nabla_{(a} \xi_{b)} = 0,
        \end{equation}
        %
        трижды проведем циклирование по индексам
        %
        \begin{equation}\begin{aligned}
            \nabla_c \nabla_{(a} \xi_{b)} &= 0 \\
            \nabla_a \nabla_{(b} \xi_{c)} &= 0 \\
            \nabla_b \nabla_{(c} \xi_{a)} &= 0
        \end{aligned}\end{equation}
        %
        Сложим первое и второе уравнения, третье вычтем. Получим
        %
        \begin{equation}\begin{aligned}
            \nabla_c \nabla_{(a} \xi_{b)} &+
            \nabla_a \nabla_{(b} \xi_{c)} -
            \nabla_b \nabla_{(c} \xi_{a)} \\
                &=  \qty( \nabla_c \nabla_a + \nabla_a \nabla_c ) \xi_b +
                    \qty( \nabla_c \nabla_b - \nabla_b \nabla_c ) \xi_a +
                    \qty( \nabla_a \nabla_b - \nabla_b \nabla_a ) \xi_c \\
                &= 2 \nabla_a \nabla_c \xi_b + \qty(
                    R_{cab}^{\hphantom{cab}q} +
                    R_{cba}^{\hphantom{cba}q} +
                    R_{abc}^{\hphantom{abc}q}
                ) \xi_q \\
                &= 0 .
        \end{aligned}\end{equation}
        %
        Применяя тождество Бьянки,
        %
        \begin{equation}
            R_{abcd} + R_{acdb} + R_{adbc} = 0,
        \end{equation}
        %
        а также правила перестановки индексов тензора кривизны,
        %
        \begin{equation}\begin{aligned}
            R_{ab,cd} &= - R_{ba,cd} = - R_{ab,dc}, \\
            R_{ab,cd} &= - R_{cd,ab}, \\
        \end{aligned}\end{equation}
        %
        получим в скобках:
        %
        \begin{equation}\begin{aligned}
            R_{cab}^{\hphantom{cab}q} +
            R_{cba}^{\hphantom{cba}q} +
            R_{abc}^{\hphantom{abc}q}
                &= g^{qd} \qty(
                    R_{cabd} + R_{cbad} + R_{abcd}
                ) \\
                &= g^{qd} \qty(
                    R_{acdb} - R_{adbc} + R_{abcd}
                ) \\
                &= g^{qd} \qty(
                    - 2 R_{adbc} + \qty{ R_{acdb} + R_{adbc} + R_{abcd} }
                ) \\
                &= - 2 g^{qd} R_{adbc} \\
                &= - 2 g^{qd} R_{bcad} \\
                &= - 2 R_{bca}^{\hphantom{bca}d} ,
        \end{aligned}\end{equation}
        %
        откуда окончательно получаем систему уравнений на векторы Киллинга:
        %
        \begin{equation}\label{eq:killing_eq}
            \nabla_a \nabla_b \xi_c = - R_{cba}^{\hphantom{cba}q} \xi_q .
        \end{equation}

        Сделаем еще один шаг. Свернем полученное выражение с метрическим тензором $g^{ab}$:
        %
        \begin{equation}\begin{aligned}
            g^{ab} \nabla_a \nabla_b \xi_c
                &= - g^{ab} R_{cba}^{\hphantom{cba}q} \xi_q \\
                &= - g^{ab} R_{cbaq} \xi^q \\
                &= - g^{ab} R_{bcqa} \xi^q \\
                &= - R_{bcq}^{\hphantom{bcq}b} \xi^q \\
                &= - R_{cq} \xi^q .
        \end{aligned}\end{equation}
        %
        Перейдем к единообразной записи в контравариантном виде, подняв индекс у $\xi_c$:
        %
        \begin{equation}\begin{aligned}\label{eq:killing_eq_lap}
            g^{ab} \nabla_a \nabla_b \xi^c
                &\equiv \nabla^i \nabla_i \xi^c
                &\equiv \Delta \xi^c
                &= - R^c_q \xi^q .
        \end{aligned}\end{equation}
        %
        Данное соотношение будет полезно при последующих доказательствах.

        Нас интересует решение \autoref{eq:killing_eq} в евклидовом пространстве. Дальнейшие выкладки приобретают громоздкий характер в случае произвольных связностей, так что пока мы ограничимся решением полученного уравнения в аффинных (декартовых) координатах, где все связности равны нулю, а метрический тензор диагонален.

        В описанном случае ковариантные производные переходят в частные, тензор кривизны обращается в нуль, и мы получаем уравнения на векторы Киллинга в более простой форме:
        %
        \begin{equation}
            \frac{\partial^2 \xi_c}{\partial x^a \partial x^b} = 0,
        \end{equation}
        %
        откуда, очевидно, любой вектор Киллинга должен быть представим в виде
        %
        \begin{equation}
            \xi_c = m_{ci} x^i + r_c,
        \end{equation}
        %
        причем $m_{ci} = - m_{ic}$ (следует из исходного уравнения на векторы Киллинга) и не зависит от $x$, в чем можно убедиться непосредственной подстановкой.

        Дальнейшие рассуждения можно свести к следующим. Будем искать простейший вид векторов Киллинга, где все $|m_{ci}|$ и $|r_c|$ равны нулю или единице, причем для конкретного вектора либо $m_{ci} = 0$, либо $r_c = 0$. Двигаясь в этом ключе, выберем самый простой вид $r^{(i)}_c = \delta^i_c$, где $(i)$ нумерует вектор Киллинга и пробегает значения $i = 1,2,3$. Также выберем три простейших по виду ортогональных матрицы $m^{(i)}_{cj} = \varepsilon^{\{i}_{jk\}}$, где за $\{ijk\}$ обозначена одна из трех возможных нечетных перестановок индексов $i,j,k$.

        Выпишем явный вид всех полученных векторов Киллинга:
        %
        \begin{equation}
            \V{\xi}_i
            =
            \left[
                \V{n}_x, \V{n}_y, \V{n}_z,
                \V{l}_x, \V{l}_y, \V{l}_z
            \right]
            =
            \begin{bmatrix}
                1 & 0 & 0 & 0  & -z & y  \\
                0 & 1 & 0 & z  & 0  & -x \\
                0 & 0 & 1 & -y & x  & 0
            \end{bmatrix}
        \end{equation}

        Из инвариантности определения вектора (тензора) следует, что и в любой другой системе координат полученные векторы будет являться полями Киллинга.

    %
    %
    %
    %%%%%%%%%%%%%%%%%%%%%%%%%%%%%%%%%%%%%%%%%%%%%%%%%%%%%%%%%%%%%%%%%%%%%%%
    %                           SECTION                                   %
    %%%%%%%%%%%%%%%%%%%%%%%%%%%%%%%%%%%%%%%%%%%%%%%%%%%%%%%%%%%%%%%%%%%%%%%
    %
    %
    %

    \section{Дифференциальные операторы и их коммутаторы\label{sec:commutators}}

        Линейным дифференциальным оператором называется конструкция вида \cite{differential_operator_commutators}:
        %
        \begin{equation}\label{eq:differential_operator}
            \D_a\tau = a^{i_1 \dots i_p} \nabla_{i_1} \dots \nabla_{i_p} \tau ,
        \end{equation}
        %
        где $\tau$ -- некоторое дифференцируемое тензорное поле. $a^{i_1 \dots i_p}$ называется порождающим тензорным полем дифференциального оператора.

        Если в \autoref{eq:differential_operator} заменить $\nabla_k$ на $\pdv*{x^k} \equiv \partial_k$, то получим другой класс дифференциальных операторов, которые будем обозначать $\partial_a\tau$ и называть производной по направлению.

        Важным является вопрос коммутации дифференциальных операторов. Два оператора коммутируют, если их коммутатор равен нулю:
        %
        \begin{equation}
            [\D_a, \D_b]\tau = (\D_a \D_b - \D_b \D_a) \tau = 0 .
        \end{equation}
        %
        В противном случае говорят, что операторы не коммутируют.

        Коммутация операторов влечет простые следствия -- у коммутирующих операторов собственные функции являются общими, что позволяет разбить задачу поиска собственных функций оператора высокого порядка на ряд задач поиска собственных функций операторов меньших порядков.

        Рассмотрим несколько важных случаев. Поскольку нам важны векторные поля, все выкладки будем проделывать применительно к ним, если особо не оговорено иного.

        %
        %
        %
        %%%%%%%%%%%%%%%%%%%%%%%%%%%%%%%%%%%%%%%%%%%%%%%%%%%%%%%%%%%%%%%%%%%
        %                        SUBSECTION                               %
        %%%%%%%%%%%%%%%%%%%%%%%%%%%%%%%%%%%%%%%%%%%%%%%%%%%%%%%%%%%%%%%%%%%
        %
        %
        %

        \subsection{Коммутация операторов первого порядка между собой}

            Для дифференциальных операторов первого порядка получаем:
            %
            \begin{equation}\begin{aligned}
                \left[ \D_\xi, \D_\eta \right] a^k
                    &= (\xi^i \nabla_i \eta^j - \eta^i \nabla_i \xi^j) \circ \nabla_j a^k \\
                    &= (\xi^i \nabla_i \eta^j - \eta^i \nabla_i \xi^j) \nabla_j a^k
                        + \xi^i \eta^j (\nabla_i \nabla_j - \nabla_j \nabla_i) a^k \\
                    &= \left[ \xi, \eta \right]^j \nabla_j a^k
                        + \xi^i \eta^j R_{ijq}^{\hphantom{ijqk}} a^q
            \end{aligned}\end{equation}
            %
            С помощью \enquote{$\circ$} явно подчеркнут операторный характер выражения в скобках. Коммутатор векторных полей понимается в смысле \autoref{eq:vector_field_commutator}.

            Таким образом, в общем случае ковариантные дифференциальные операторы первого порядка не коммутируют. Их коммутатор имеет наиболее простой вид, если порождающие их векторные поля коммутируют в смысле \autoref{eq:vector_field_commutator}. Если кроме этого тензор кривизны всюду в пространстве равен нулевому тензору, операторы первого порядка коммутируют.

            Более простое выражение получается для коммутатора производных по направлению первого порядка:
            %
            \begin{equation}\begin{aligned}
                \left[ \partial_\xi, \partial_\eta \right] a^q
                    &= (\xi^i \partial_i \eta^j - \eta^i \partial_i \xi^j) \circ \partial_j a^q \\
                    &= (\xi^i \partial_i \eta^j - \eta^i \partial_i \xi^j) \partial_j a^q
                        + \xi^i \eta^j (\partial_i \partial_j - \partial_j \partial_i) a^q \\
                    &= \left[ \xi, \eta \right]^j \partial_j a^q .
            \end{aligned}\end{equation}
            %
            Как и следовало ожидать, полученное выражение не зависит от кривизны пространства, поскольку характеристики пространства при \enquote{обычном} дифференцировании не учитываются.

        %
        %
        %
        %%%%%%%%%%%%%%%%%%%%%%%%%%%%%%%%%%%%%%%%%%%%%%%%%%%%%%%%%%%%%%%%%%%
        %                        SUBSECTION                               %
        %%%%%%%%%%%%%%%%%%%%%%%%%%%%%%%%%%%%%%%%%%%%%%%%%%%%%%%%%%%%%%%%%%%
        %
        %
        %

        \subsection{Коммутация операторов Киллинга между собой и с дифференциальными операторами первого порядка}

            Покажем на примере векторного поля, что для коммутации операторов Киллинга необходимо, чтобы порождающие их векторные поля Киллинга коммутировали. Для этого представим оператор Киллинга в виде двух операторов:
            %
            \begin{equation}\begin{aligned}
                \Op{\xi} a^k
                    &= \xi^c \nabla_c a^k - a^c \nabla_c \xi^k \\
                    &= \qty(\D_\xi - \D^{*}_\xi) a^k .
            \end{aligned}\end{equation}
            %
            Раскроем коммутатор по правилу Лейбница:
            %
            \begin{equation}\begin{aligned}
                \comm{\Op{\xi}}{\Op{\eta}} a^k
                    &= \comm{\D_\xi - \D^{*}_\xi}{\D_\eta - \D^{*}_\eta} a^k \\
                    &= \comm{\D_\xi}{\D_\eta} a^k
                     + \comm{\D^{*}_\xi}{\D^{*}_\eta} a^k
                     - \comm{\D^{*}_\xi}{\D_\eta} a^k
                     - \comm{\D_\xi}{\D^{*}_\eta} a^k .
            \end{aligned}\end{equation}
            %
            При вычислении коммутаторов для сокращения объема выкладок сразу будем полагать, что $\comm{\xi}{\eta} = 0$.

            Выражение для первого коммутатора было получено выше, поэтому раскроем следующий коммутатор:
            %
            \begin{equation}\begin{aligned}
                \comm{\D^{*}_\xi}{\D_\eta} a^k
                    &= \D^{*}_\xi \D_\eta a^k - \D_\eta \D^{*}_\xi a^k \\
                    &= \D^{*}_\xi \qty( \eta^c \nabla_c a^k )
                     - \D_\eta \qty( a^c \nabla_c \xi^k ) \\
                    &= \qty( \eta^c \nabla_c a^d ) \nabla_d \xi^k
                     - \eta^d \nabla_d \qty( a^c \nabla_c \xi^k ) \\
                    &= \qty( \eta^c \nabla_c a^d ) \nabla_d \xi^k
                     - \qty( \eta^d \nabla_d a^c ) \nabla_c \xi^k
                     - \eta^d a^c \nabla_d \nabla_c \xi^k \\
                    &= - \eta^d a^c \nabla_d \nabla_c \xi^k .
            \end{aligned}\end{equation}
            %
            Аналогично,
            %
            \begin{equation}\begin{aligned}
                \comm{\D_\xi}{\D^{*}_\eta} a^k
                    &= \xi^d a^c \nabla_d \nabla_c \eta^k .
            \end{aligned}\end{equation}
            %
            Наконец,
            %
            \begin{equation}\begin{aligned}
                \comm{\D^{*}_\xi}{\D^{*}_\eta} a^k
                    &= \D^{*}_\xi \D^{*}_\eta a^k - \D^{*}_\eta \D^{*}_\xi a^k \\
                    &= \D^{*}_\xi \qty( a^c \nabla_c \eta^k )
                     - \D^{*}_\eta \qty( a^c \nabla_c \xi^k ) \\
                    &= \qty( a^c \nabla_c \eta^d ) \nabla_d \xi^k
                     - \qty( a^c \nabla_c \xi^d ) \nabla_d \eta^k \\
                    &= a^c \qty{
                        \nabla_c \qty( \eta^d \nabla_d \xi^k - \xi^d \nabla_d \eta^k ) -
                        \qty( \eta^d \nabla_c \nabla_d \xi^k - \xi^d \nabla_c \nabla_d \eta^k )
                    } \\
                    &= - \eta^d a^c \nabla_c \nabla_d \xi^k + \xi^d a^c \nabla_c \nabla_d \eta^k .
            \end{aligned}\end{equation}
            %
            Теперь, соединяя все воедино, получим:
            %
            \begin{equation}\begin{aligned}
                \comm{\Op{\xi}}{\Op{\eta}} a^k
                    &= \comm{\D_\xi}{\D_\eta} a^k
                     + \comm{\D^{*}_\xi}{\D^{*}_\eta} a^k
                     - \comm{\D^{*}_\xi}{\D_\eta} a^k
                     - \comm{\D_\xi}{\D^{*}_\eta} a^k \\
                    &= \xi^c \eta^d R_{cda}^{\hphantom{cdak}} a^a
                     - \eta^d a^c \nabla_c \nabla_d \xi^k
                     + \xi^d a^c \nabla_c \nabla_d \eta^k
                     + \eta^d a^c \nabla_d \nabla_c \xi^k
                     - \xi^d a^c \nabla_d \nabla_c \eta^k \\
                    &= \xi^c \eta^d a^a R_{cda}^{\hphantom{cdak}}
                     - \xi^a \eta^d a^c R_{cda}^{\hphantom{cdak}}
                     + \xi^d \eta^a a^c R_{cda}^{\hphantom{cdak}} \\
                    &= \xi^c \eta^d a^a g^{kb} \qty(
                        R_{cdab} - R_{adcb} + R_{acdb}
                    ) \\
                    &= \xi^c \eta^d a^a g^{kb} \qty(
                        R_{abcd} + R_{adbc} + R_{acdb}
                    ) \\
                    &= 0 .
            \end{aligned}\end{equation}

            Исходя из этого, формулируется предложение: два оператора Киллинга коммутируют, если коммутируют порождающие их векторные поля. Это предложение также остается справедливым \cite{burlankov_space_dynamics}, если рассматривать операторы Киллинга на полях произвольной тензорной размерности.

        %
        %
        %
        %%%%%%%%%%%%%%%%%%%%%%%%%%%%%%%%%%%%%%%%%%%%%%%%%%%%%%%%%%%%%%%%%%%
        %                        SUBSECTION                               %
        %%%%%%%%%%%%%%%%%%%%%%%%%%%%%%%%%%%%%%%%%%%%%%%%%%%%%%%%%%%%%%%%%%%
        %
        %
        %

        \subsection{Коммутационные соотношения оператора Лапласа}

            Для оператора Лапласа коммутационные соотношения принимают громоздкий характер. Уже для коммутатора $\comm{\Delta}{\D_\xi} a^k$ выражение содержит тензор кривизны и другие неисчезающие члены. Коммутатор $\comm{\Delta}{\Op{\xi}} a^k$ также не обращается в нуль, а содержит в т.ч. и неисчезающие производные от тензора кривизны. Для дальнейших рассуждений настолько общие выкладки нам не понадобятся, поэтому теоретическое выяснение общих коммутационных соотношений дифференциальных операторов мы на этом закончим.

            Следует, однако, отметить, что для случая скалярных полей для $[\Delta, D_\xi] f = 0$ необходимо, чтобы $\xi$ являлось полем Киллинга \cite{differential_operator_commutators}.

    %
    %
    %
    %%%%%%%%%%%%%%%%%%%%%%%%%%%%%%%%%%%%%%%%%%%%%%%%%%%%%%%%%%%%%%%%%%%%%%%
    %                           SECTION                                   %
    %%%%%%%%%%%%%%%%%%%%%%%%%%%%%%%%%%%%%%%%%%%%%%%%%%%%%%%%%%%%%%%%%%%%%%%
    %
    %
    %

    \section{Коммутационные соотношения операторов Киллинга}

        Найдем коммутационные соотношения внутри и между группами движений евклидова пространства:
        %
        \begin{equation}
            \begin{bmatrix}
                [ \Op{n}_x \Op{n}_y ] & [ \Op{n}_z \Op{n}_x ] & [ \Op{n}_y \Op{n}_z ] \\
                [ \Op{l}_x \Op{l}_y ] & [ \Op{l}_z \Op{l}_x ] & [ \Op{l}_y \Op{l}_z ] \\
                [ \Op{n}_x \Op{l}_x ] & [ \Op{n}_y \Op{l}_x ] & [ \Op{n}_z \Op{l}_x ] \\
                [ \Op{n}_x \Op{l}_y ] & [ \Op{n}_y \Op{l}_y ] & [ \Op{n}_z \Op{l}_y ] \\
                [ \Op{n}_x \Op{l}_z ] & [ \Op{n}_y \Op{l}_z ] & [ \Op{n}_z \Op{l}_z ]
            \end{bmatrix}
            =
            \begin{bmatrix}
                0          &   0        &   0        \\
                  \Op{l}_z &   \Op{l}_y &   \Op{l}_x \\
                0          &   \Op{n}_z & - \Op{n}_y \\
                - \Op{n}_z &   0        &   \Op{n}_x \\
                  \Op{n}_y & - \Op{n}_x &   0
            \end{bmatrix}
        \end{equation}
        %

        Полученные коммутаторы инвариантны относительно выбора системы координат.

        Составим оператор
        %
        \begin{equation}
            \Op{l}^2 = \Op{l}_x \Op{l}_x + \Op{l}_y \Op{l}_y + \Op{l}_z \Op{l}_z .
        \end{equation}
        %
        Очевидно, он коммутирует с любыми из операторов вращений $\Op{l}_i$, в чем можно убедиться непосредственной проверкой. Данный оператор будет полезен нам в дальнейших выкладках.

    %
    %
    %
    %%%%%%%%%%%%%%%%%%%%%%%%%%%%%%%%%%%%%%%%%%%%%%%%%%%%%%%%%%%%%%%%%%%%%%%
    %                           SECTION                                   %
    %%%%%%%%%%%%%%%%%%%%%%%%%%%%%%%%%%%%%%%%%%%%%%%%%%%%%%%%%%%%%%%%%%%%%%%
    %
    %
    %

    \section{Коммутационные соотношения оператора Лапласа и операторов Киллинга}

        В декартовой системе координат оператор Лапласа имеет простой вид:
        %
        \begin{equation}
            \Delta = \pdv[2]{x} + \pdv[2]{y} + \pdv[2]{z} .
        \end{equation}
        %
        Его можно разложить по операторам Киллинга сдвигов, $\Op{n}_x$, $\Op{n}_y$, $\Op{n}_z$:
        %
        \begin{equation}
            \Delta = \Op{n}_x\Op{n}_x + \Op{n}_y\Op{n}_y + \Op{n}_z\Op{n}_z .
        \end{equation}
        %
        Откуда нетрудно видеть, что оператор Лапласа коммутирует со всеми операторами вращений и сдвигов, причем эти коммутационные соотношения не будут зависеть от выбора координатной системы.

        Вернемся к оператору $\Op{l}^2$. Покажем, что он коммутирует с оператором Лапласа (применено правило Лейбница и тот факт, что оператор Лапласа коммутирует с любыми из операторов поворота):
        %
        \begin{equation}\begin{aligned}
            \left[ \Delta, \Op{l}^2 \right]
                &= \left[ \Delta, \Op{l}_x \Op{l}_x \right]
                    + \left[ \Delta, \Op{l}_y \Op{l}_y \right]
                    + \left[ \Delta, \Op{l}_z \Op{l}_z \right] \\
                &= \left[ \Delta, \Op{l}_x \right] \Op{l}_x
                    + \Op{l}_x \left[ \Delta, \Op{l}_x \right]
                    + \left[ \Delta, \Op{l}_y \right] \Op{l}_y
                    + \Op{l}_y \left[ \Delta, \Op{l}_y \right]
                    + \left[ \Delta, \Op{l}_z \right] \Op{l}_z
                    + \Op{l}_z \left[ \Delta, \Op{l}_z \right] \\
                &= 0
        \end{aligned}\end{equation}

    %
    %
    %
    %%%%%%%%%%%%%%%%%%%%%%%%%%%%%%%%%%%%%%%%%%%%%%%%%%%%%%%%%%%%%%%%%%%%%%%
    %                           SECTION                                   %
    %%%%%%%%%%%%%%%%%%%%%%%%%%%%%%%%%%%%%%%%%%%%%%%%%%%%%%%%%%%%%%%%%%%%%%%
    %
    %
    %

    \section{Поля Киллинга в сферической системе координат}

        Из инвариантности объекта \enquote{вектор}, задающего направление Ли-вариации, следует, что поля Киллинга, найденные в одной системе координат, останутся полями Киллинга и в другой системе координат. Найдем векторы Киллинга в сферической координатной системе с метрикой
        %
        \begin{equation}
            dl^2 = dr^2 + r^2 (d\theta^2 + \sin ^2 \theta d\varphi^2)
        \end{equation}
        %
        и связью между координатами декартовой и сферической системами координат
        %
        \begin{equation}
            \V{x}(r, \theta, \varphi)
            =
            \{x^i\}
            =
            \begin{bmatrix}
                x \\ y \\ z
            \end{bmatrix}
            =
            \begin{bmatrix}
                r \sin\theta \cos\varphi \\
                r \sin\theta \sin\varphi \\
                r \cos\theta
            \end{bmatrix}
            .
        \end{equation}
        %
        Для чего перейдем к новой системе координат:
        %
        \begin{equation}\begin{aligned}
            \xi_{i'} = \frac{\partial x^j}{\partial x^{i'}} \xi_j.
        \end{aligned}\end{equation}
        %
        Проведенные вычисления дают
        %
        \begin{equation}
            \V{n}_i
            =
            \begin{bmatrix}
                \sin\theta \cos\varphi     & \sin\theta \sin\varphi   & \cos\theta \\
                r \cos\theta \cos\varphi   & r \cos\theta \sin\varphi & - r \sin\theta \\
                - r \sin\theta \sin\varphi & r \sin\theta \cos\varphi & 0
            \end{bmatrix}
            ,
        \end{equation}
        %
        \begin{equation}
            \V{l}_i
            =
            \begin{bmatrix}
                0
                    & 0
                    & 0 \\
                r^2 \sin\varphi
                    & - r^2 \cos\varphi
                    & 0 \\
                r^2 \cos\varphi \cos\theta \sin\theta
                    & r^2 \sin\varphi \cos\theta \sin\theta
                    & - r^2 \sin^2\theta
            \end{bmatrix}
            .
        \end{equation}
        %
        Приведенные соотношения, однако, выглядят проще в контравариантном виде:
        %
        \begin{equation}
            \V{n}_i
            =
            \begin{bmatrix}
                \sin\theta \cos\varphi          & \sin\theta \sin\varphi        & \cos\theta \\
                r^{-1} \cos\theta \cos\varphi   & r^{-1} \cos\theta \sin\varphi & - r^{-1} \sin\theta \\
                - r^{-1} \csc\theta \sin\varphi & r^{-1} \csc\theta \cos\varphi & 0
            \end{bmatrix}
            ,
        \end{equation}
        %
        \begin{equation}
            \V{l}_i
            =
            \begin{bmatrix}
                0
                    & 0
                    & 0 \\
                \sin\varphi
                    & \cos\varphi
                    & 0 \\
                \cos\varphi \cot\theta
                    & \cot\theta \sin\varphi
                    & - 1
            \end{bmatrix}
            .
        \end{equation}

    %
    %
    %
    %%%%%%%%%%%%%%%%%%%%%%%%%%%%%%%%%%%%%%%%%%%%%%%%%%%%%%%%%%%%%%%%%%%%%%%
    %                           SECTION                                   %
    %%%%%%%%%%%%%%%%%%%%%%%%%%%%%%%%%%%%%%%%%%%%%%%%%%%%%%%%%%%%%%%%%%%%%%%
    %
    %
    %

    \section{Собственные функции и собственные значения операторов вращений Киллинга}

        Найдем связь между собственными значениями и собственными функциями операторов Киллинга вращений. Будем называть собственные функции модами, причем природу этих функций (скалярное, векторное, тензорное поле) мы пока не конкретизуем.

        Введем операторы $\Op{l}_{+} = \Op{l}_x - i\Op{l}_y$ и $\Op{l}_{-} = \Op{l}_x + i\Op{l}_y$ \cite{burlankov_space_dynamics,burlankov_tmf}. Коммутационные соотношения между ними и оператором $\Op{l}_z$:
        %
        \begin{equation}
            \left[ \Op{l}_{+}, \Op{l}_{-} \right] = 2 i \Op{l}_{z} ;
            \left[ \Op{l}_{z}, \Op{l}_{+} \right] =   i \Op{l}_{+} ;
            \left[ \Op{l}_{z}, \Op{l}_{-} \right] = - i \Op{l}_{-} .
        \end{equation}
        %

        Пусть мода $h_m$ является собственной функцией оператора $\Op{l}_z$ с собственным значением $i m$. Тогда мода $h_{+} = \Op{l}_{+} h_m$ тоже будет являться собственной функцией оператора $\Op{l}_z$ с собственным значением $i (m + 1)$: $h_{+} = h_{m + 1}$. Аналогично, $h_{-} = h_{m - 1}$. В этом можно убедиться непосредственной проверкой. В силу этого $\Op{l}_{+}$ и $\Op{l}_{-}$ называют операторами повышения и понижения.

        Пусть теперь мода $g_\lambda$ является является собственной функцией оператора $\Op{l}^2$ с собственным значением $\lambda$. Поскольку оператор $\Op{l}^2$ коммутирует с $\Op{l}_z$, мода $g_\lambda$ также является и собственной функцией оператора $\Op{l}_z$, а значит можно обозначить $g_\lambda = h_m = h_{\lambda,m}$.

        Найдем связь между $\lambda$ и $m$. Максимальное значение $m$, т.е. значение, при котором $\Op{l}_{+} h_{\lambda,m} = 0$, обозначим $l$. Представим оператор $\Op{l}^2$ в виде
        %
        \begin{equation}\begin{aligned}
            \Op{l}^2
                &= \Op{l}_{-}\Op{l}_{+} + \Op{l}^2_{z} + i \Op{l}_{z} \\
                &= \Op{l}_{+}\Op{l}_{-} + \Op{l}^2_{z} - i \Op{l}_{z} .
        \end{aligned}\end{equation}
        %
        Подействуем оператором $\Op{l}_{-}\Op{l}_{+}$ на моду $h_{\lambda,l}$:
        %
        \begin{equation}\begin{aligned}
            \Op{l}_{-}\Op{l}_{+} h_{\lambda,l}
                &= (\Op{l}^2 - \Op{l}^2_{z} - i \Op{l}_{z}) h_{\lambda,l} \\
                &= (\lambda + l^2 + l) h_{\lambda,l} \\
                &= 0 ,
        \end{aligned}\end{equation}
        %
        откуда следует, что $\lambda = - l (l + 1)$. Применяя другой оператор, $\Op{l}_{+}\Op{l}_{-}$, мы увидим, что минимальное значение $m = - l$.

    %
    %
    %
    %%%%%%%%%%%%%%%%%%%%%%%%%%%%%%%%%%%%%%%%%%%%%%%%%%%%%%%%%%%%%%%%%%%%%%%
    %                        BIBLIOGRAPHY                                 %
    %%%%%%%%%%%%%%%%%%%%%%%%%%%%%%%%%%%%%%%%%%%%%%%%%%%%%%%%%%%%%%%%%%%%%%%
    %
    %
    %

    \nocite{*}
    \bibliographystyle{../../lib/doc/bib/utf8gost705s}
    \bibliography{../../lib/doc/bib/math,../../lib/doc/bib/physics}

\end{document}
