\documentclass[12pt,a4paper]{article}
\usepackage[T1,T2A]{fontenc}
\usepackage[utf8]{inputenc}
\usepackage[english,russian]{babel}
\usepackage{microtype}
\usepackage{csquotes}
\usepackage{amsmath}
\usepackage{amsthm}
\usepackage{amssymb}
\usepackage{mathtext}
\usepackage{newfloat}
\usepackage{caption}
\usepackage{indentfirst}
\usepackage{geometry}
\usepackage{hyperref}
\usepackage{mdframed}
\usepackage[inline]{enumitem}
\usepackage{graphicx}
\usepackage{subfig}
\usepackage[titletoc,toc]{appendix}

\DeclareGraphicsExtensions{.pdf,.png,.jpg,.PNG}
\graphicspath{{./img/}}
\captionsetup[figure]{justification=centering}
\renewcommand{\thesubfigure}{\asbuk{subfigure}}
\DeclareCaptionLabelSeparator{dotseparator}{. }
\captionsetup{labelsep=dotseparator}
\geometry{left=1cm,right=2cm,top=2cm,bottom=2cm}
\makeatletter\appto{\appendices}{\def\Hy@chapapp{Appendix}}\makeatother
\renewcommand{\appendixtocname}{Приложения}
\renewcommand{\appendixpagename}{Приложения}

\DeclareMathOperator{\Rot}{\mathbf{rot}}
\DeclareMathOperator{\Grad}{\mathbf{grad}}
\DeclareMathOperator{\Div}{\mathbf{div}}
\DeclareMathOperator{\D}{D}
\newcommand{\V}[1]{\mathbf{#1}}

\title{Математическое приложение}
\author{Василевский А.В.}

\begin{document}

    \maketitle
    \tableofcontents

    \appendix

    %
    %
    %
    %%%%%%%%%%%%%%%%%%%%%%%%%%%%%%%%%%%%%%%%%%%%%%%%%%%%%%%%%%%%%%%%%%%%%%%
    %                           SECTION                                   %
    %%%%%%%%%%%%%%%%%%%%%%%%%%%%%%%%%%%%%%%%%%%%%%%%%%%%%%%%%%%%%%%%%%%%%%%
    %
    %
    %

    \section{Волновое уравнение}

        Получим стационарные уравнения Максвелла в свободном пространстве для монохроматической волны в нерелятивистском приближении.

        Исходная система уравнений имеет вид

        \begin{equation}\begin{aligned}\label{eq:maxwell_empty_space}
            \Rot\V{E} &= - \frac{1}{c} \frac{\partial \V{B}}{\partial t}, \\
            \Rot\V{B} &= \frac{1}{c} \frac{\partial \V{E}}{\partial t}, \\
            \Div\V{B} &= 0, \\
            \Div\V{E} &= 0.
        \end{aligned}\end{equation}
        %
        При
        %
        \begin{equation}\begin{aligned}
            \V{E}(\V{r}, t) &= \V{\hat{E}}(\V{r}) \exp(- i \omega t), \\
            \V{B}(\V{r}, t) &= \V{\hat{B}}(\V{r}) \exp(- i \omega t)
        \end{aligned}\end{equation}
        %
        уравнения принимают вид
        %
        \begin{equation}\begin{aligned}
            \Rot\V{\hat{E}} &= i \omega \frac{1}{c} \V{\hat{B}}, \\
            \Rot\V{\hat{B}} &= - i \omega \frac{1}{c} \V{\hat{E}}.
        \end{aligned}\end{equation}
        %
        Заметим, что вторые два уравнения удовлетворяются в автоматически. Взятием ротора от первого уравнения и подстановкой в него второго уравнения системы можно получить следующее выражение для вектора $\V{E}$ (аналогичное получается и для $\V{H}$):
        %
        \begin{equation}
            \Rot\Rot{\V{\hat{E}}} = i \omega \frac{1}{c} \left( -i \omega \frac{1}{c} \V{\hat{E}} \right) = \frac{\omega^2}{c^2} \V{\hat{E}}.
        \end{equation}
        %
        Полученное уравнение называется стационарным волновым уравнением.

        Его можно упростить, принимая во внимание вторую пару уравнений \autoref{eq:maxwell_empty_space} и тождество
        %
        \begin{equation}\label{eq:laplacian_vect}
            \Rot\Rot = \Grad\Div - \Delta,
        \end{equation}
        %
        где $\Delta$ -- векторный оператор Лапласа:
        %
        \begin{equation}
            \Delta\V{\hat{E}} = - \frac{\omega^2}{c^2} \V{\hat{E}}.
        \end{equation}

        В дальнейшем условимся опускать знак \enquote{$\hat{\ }$}, имея в виду именно не зависящую от времени часть $\V{E}$, если не сказано обратного.

    %
    %
    %
    %%%%%%%%%%%%%%%%%%%%%%%%%%%%%%%%%%%%%%%%%%%%%%%%%%%%%%%%%%%%%%%%%%%%%%%
    %                           SECTION                                   %
    %%%%%%%%%%%%%%%%%%%%%%%%%%%%%%%%%%%%%%%%%%%%%%%%%%%%%%%%%%%%%%%%%%%%%%%
    %
    %
    %

    \section{Векторный оператор Лапласа в произвольных координатах}

        Ограничимся трехмерным евклидовым пространством. Получим явный инвариантный вид введенного оператора Лапласа. Будем опираться на формулировку \autoref{eq:laplacian_vect}.

        Цель данного пункта -- показать, что введенный выше векторный оператор Лапласа не является особым объектом, принадлежит к классу дифференциальных операторов второго порядка, а его вид в координатной записи не зависит от поля, к которому он применяется.

        Ротор в криволинейных координатах через ковариантные производные от ковариантных компонент вектора $\V{a}$ представлен в виде
        %
        \begin{equation}
            \left( \Rot\V{a} \right)^i
                = \frac{1}{2} \varepsilon^{ijk} \left(
                    \nabla_k a_j - \nabla_j a_k
                \right)
                \equiv \varepsilon^{ijk} \nabla_{[k} a_{j]}.
        \end{equation}
        %
        Распишем $\Rot\Rot\V{a}$:
        %
        \begin{equation}
            \left( \Rot\Rot\V{a} \right)^i
                = \varepsilon^{ijk} \nabla_{[k} \left( \Rot\V{a} \right)_{j]}.
        \end{equation}
        %
        Получим явный вид альтернированной ковариантной производной ротора:
        %
        \begin{equation}\begin{aligned}
            \nabla_{[k} \left( \Rot\V{a} \right)_{j]}
                &= \nabla_{[k} \left( g_{j]m} \left( \Rot\V{a} \right)^m \right) \\
                &= g_{[jm} \nabla_{k]} \left( \Rot\V{a} \right)^m \\
                &= g_{[jm} \nabla_{k]} \left(
                       \varepsilon^{mqr} \nabla_{[r} a_{q]}
                \right) \\
                &= g_{[jm} \varepsilon^{mqr} \left(
                   \nabla_{k]} \nabla_{[r} a_{q]}
                \right),
        \end{aligned}\end{equation}
        %
        откуда
        %
        \begin{equation}
            \left( \Rot\Rot\V{a} \right)^i
                = \varepsilon^{ijk} g_{[jm} \varepsilon^{mqr} \left(
                    \nabla_{k]} \nabla_{[r} a_{q]}
                \right).
        \end{equation}
        %
        Распишем произведение перед скобкой, пользуясь тождеством
        %
        \begin{equation}
           \varepsilon_{ijk} \varepsilon^{imn} = \delta_j^m \delta_k^n - \delta_j^n \delta_k^m,
        \end{equation}
        %
        получим
        %
        \begin{equation}\begin{aligned}
           \varepsilon^{ijk} g_{jm} \varepsilon^{mqr}
                &= \varepsilon^{kij} g_{jm} \varepsilon^{mqr} \\
                &= g^{ip} g^{ks} \varepsilon_{msp} \varepsilon^{mqr} \\
                &= g^{ip} g^{ks} \left(
                    \delta_s^q \delta_p^r - \delta_s^r \delta_p^q
                \right)
        \end{aligned}\end{equation}
        %
        и аналогично для альтернированного по индексам $[jk]$ варианта
        %
        \begin{equation}\begin{aligned}
           \varepsilon^{ijk} g_{km} \varepsilon^{mqr}
                &= \varepsilon^{ijk} g_{jm} \varepsilon^{mqr} \\
                &= g^{ip} g^{js} \varepsilon_{mps} \varepsilon^{mqr} \\
                &= g^{ip} g^{js} \left(
                    \delta_p^q \delta_s^r - \delta_p^r \delta_s^q
                \right) \\
                &= - g^{ip} g^{js} \left(
                    \delta_s^q \delta_p^r - \delta_s^r \delta_p^q
                \right).
        \end{aligned}\end{equation}

        В итоге, в явном виде ротор ротора перепишется следующим образом:
        %
        \begin{equation}\begin{aligned}
            \left( \Rot\Rot\V{a} \right)^i
                &= \frac{1}{2} g^{ip} \left\{
                       g^{ks} \left(
                           \delta_s^q \delta_p^r - \delta_s^r \delta_p^q
                       \right) \nabla_k \nabla_{[r} a_{q]} +
                       g^{js} \left(
                           \delta_s^q \delta_p^r - \delta_s^r \delta_p^q
                       \right) \nabla_j \nabla_{[r} a_{q]}
                   \right\} \\
                &= g^{ip} \left(
                       g^{kq} \nabla_k \nabla_{[p} a_{q]} - g^{kr} \nabla_k \nabla_{[r} a_{p]}
                   \right) \\
                &= 2 g^{ip} g^{kq} \left(
                       \nabla_k \nabla_{[p} a_{q]}
                   \right) \\
                &= g^{ip} g^{kq} \left(
                       \nabla_k \nabla_{p} a_{q} - \nabla_k \nabla_{q} a_{p}
                   \right) \\
                &= g^{ip} \nabla_k \nabla_{p} a^k - g^{kq} \nabla_k \nabla_{q} a^i
        \end{aligned}\end{equation}
        %
        Пользуясь теперь
        %
        \begin{equation}\begin{gathered}
            (\nabla_k \nabla_p - \nabla_p \nabla_k) a^i = - R_{kpm}^{\hphantom{kpm}i} a^m , \\
            R_{kpm}^{\hphantom{kpm}i} \delta_i^k = R_{kpm}^{\hphantom{kpm}k} = R_{pm} ,
        \end{gathered}\end{equation}
        %
        где $R_{kpm}^{\hphantom{kpm}i}$ -- тензор кривизны (тензор Римана-Кристоффеля), а $R_{pm}$, соответственно, тензор Риччи, преобразуем последнюю строчку к виду
        %
        \begin{equation}\begin{aligned}
            \left( \Rot\Rot\V{a} \right)^i
                &= g^{ip} \nabla_k \nabla_p a^k - g^{kq} \nabla_k \nabla_q a^i \\
                &= g^{ip} \nabla_p \nabla_k a^k
                    + g^{ip} R_{pm} a^m
                    - g^{kq} \nabla_k \nabla_q a^i \\
                &= \left( \Grad\Div{\V{a}} - \Delta{\V{a}} \right)^i + g^{ip} R_{pm} a^m
        \end{aligned}\end{equation}
        %
        В пространствах с нулевой кривизной (а мы рассматриваем такие пространства) тензор Риччи тождественно равен нулю.

        Итак, мы получили общий вид оператора Лапласа в произвольной системе координат. Выпишем его отдельно:
        %
        \begin{equation}
            \Delta{\V{a}} = g^{kq} \nabla_k \nabla_{q} a^i
                          = \nabla^k \nabla_k a^i
        \end{equation}
        %
        Мы также видим, что вид оператора Лапласа не зависит от объекта, к которому он применяется (скалярное, векторное, тензорное поле). Оператор Лапласа является частным случаем более общего семейства линейных дифференциальных операторов порядка $p$, порождаемых определенным тензорным полем:
        %
        \begin{equation}
            \D^p\tau = a^{i_1 \dots i_p} \nabla_{i_1} \dots \nabla_{i_p} \tau ,
        \end{equation}
        %
        где под $\tau$ понимается тензорное поле произвольной (в т.ч. нулевой) валентности. Ниже будут рассмотрены коммутационные соотношения и условия коммутации между операторами вплоть до операторов второго порядка.

    %
    %
    %
    %%%%%%%%%%%%%%%%%%%%%%%%%%%%%%%%%%%%%%%%%%%%%%%%%%%%%%%%%%%%%%%%%%%%%%%
    %                           SECTION                                   %
    %%%%%%%%%%%%%%%%%%%%%%%%%%%%%%%%%%%%%%%%%%%%%%%%%%%%%%%%%%%%%%%%%%%%%%%
    %
    %
    %

    \section{Ли-вариация, поля, операторы Киллинга}

        Будем рассматривать, как обычно, трехмерные пространства без кручения.

        Ли-вариация является инвариантным обобщением производной по направлению на произвольные тензорные поля. Для произвольного тензорного поля $\tau(P)$ вводится оператор
        %
        \begin{equation}
            \delta_\xi \tau(P) = \lim\limits_{\varepsilon \to 0} \frac{
                \tau(P + \varepsilon \xi) - \tau(P)
            }{\varepsilon}.
        \end{equation}
        %

        Второе эквивалентное определение (\cite{lie_derivative_theory}) состоит в следующем. Рассмотрим тензорное поле $\tau(\xi)$ и бесконечно малое координатное преобразование $T: \xi \rightarrow {'\xi}$ такое, что в заданной некоторой системе $\varkappa$ $T$ выражается преобразованием $'\xi^i = \xi^i + v^i d\varepsilon$, где $v^i$ -- некоторое векторное поле. При применении преобразования координат $T^{-1}: \varkappa \rightarrow {'\varkappa}$, причем $'\xi^i \rightarrow \xi^{i'}$, мы получим тензор $'\tau^{'\varkappa}(\xi)$, заданный, однако, в $'\varkappa$. Наконец, рассмотрим тензор $'\tau(\xi)$, заданный в $\varkappa$ и имеющий те же координаты, что и полученный $'\tau^{'\varkappa}(\xi)$ в $'\varkappa$. Ли-вариацией тензорного поля $\tau(\xi)$ по векторному полю $v$ называется разность между $\tau(\xi)$ и $'\tau(\xi)$, деленная на $d\varepsilon$.

        Ли-вариация тензорного поля ${T^{a_1 \dots a_p}}_{b_1 \dots b_q}(M)$ по направлению векторного поля $\xi(M)$ в координатной записи имеет вид:
        %
        \begin{equation}\begin{aligned}
            \delta_\xi {T^{a_1 \dots a_p}}_{b_1 \dots b_q}
                &= \xi^c \left( \nabla_c {T^{a_i \dots a_p}}_{b_1 \dots b_q} \right) \\
                &- \left( \nabla_{c} \xi^{a_1} \right) {T^{c a_2 \dots a_p}}_{b_1 \dots b_q} - \dots
                 - \left( \nabla_{c} \xi^{a_p} \right) {T^{a_1 \dots a_{p-1} c}}_{b_1 \dots b_q} \\
                &+ \left( \nabla_{b_1} \xi^c \right) {T^{a_1 \dots a_p}}_{c b_2 \dots b_q} + \dots
                 + \left( \nabla_{b_q} \xi^c \right) {T^{a_1 \dots a_p}}_{b_1 \dots b_{q-1} c}
        \end{aligned}\end{equation}
        %
        В частности, для векторных полей справедливо
        %
        \begin{equation}\begin{aligned}
            \delta_\xi T^a
                = \xi^c \nabla_c T^a - T^c \nabla_{c} \xi^a
                \equiv [\xi, T]^a.
        \end{aligned}\end{equation}
        %
        Этим вводится понятие коммутатора двух векторных полей (Ли-коммутатор, скобка Ли).

        Векторные поля, на которых Ли-вариация метрического тензора равна нулю, называются движениями, или полями Киллинга:
        %
        \begin{equation}\begin{aligned}
            \delta_\xi g_{ab}
                &= \xi^c \nabla_c g_{ab} + g_{cb} \nabla_{c} \xi^a + g_{ac} \nabla_{c} \xi^a \\
                &= g_{cb} \nabla_{a} \xi^c + g_{ac} \nabla_{b} \xi^c \\
                &= \nabla_{a} \xi_b + \nabla_{b} \xi_a \\
                &= 2 \nabla_{(a} \xi_{b)}.
        \end{aligned}\end{equation}

        Найдем векторы Киллинга трехмерного евклидова пространства. Запишем условие инвариантности метрики:
        %
        \begin{equation}
            \delta_\xi g_{ab}
                = 2 \nabla_{(a} \xi_{b)}
                = 0
                \Leftrightarrow \nabla_{(a} \xi_{b)} = 0.
        \end{equation}
        %
        Продифференцируем полученное выражение
        %
        \begin{equation}
            \nabla_c \nabla_{(a} \xi_{b)} = 0,
        \end{equation}
        %
        трижды проведем циклирование по индексам
        %
        \begin{equation}\begin{aligned}
            \nabla_c \nabla_{(a} \xi_{b)} &= 0 \\
            \nabla_a \nabla_{(b} \xi_{c)} &= 0 \\
            \nabla_b \nabla_{(c} \xi_{a)} &= 0
        \end{aligned}\end{equation}
        %
        Сложим первое и второе уравнения, третье вычтем. Получим
        %
        \begin{equation}\begin{aligned}
            \nabla_c \nabla_{(a} \xi_{b)} +
            \nabla_a \nabla_{(b} \xi_{c)} -
            \nabla_b \nabla_{(c} \xi_{a)} =
            2 \nabla_a \nabla_b \xi_c     = 0,
        \end{aligned}\end{equation}
        %
        откуда получаем систему уравнений на векторы Киллинга:
        %
        \begin{equation}
            \nabla_a \nabla_b \xi_c = 0.
        \end{equation}
        %
        Полученное выражение очень похоже на требование равенства нулю оператора Лапласа $\Delta \equiv g^{ab} \nabla_a \nabla_b$, но отличается от него отсутствием свертки с метрическим тензором, что \enquote{поправимо} путем искусственной свертки данного уравнения с метрическим тензором. Отсюда видно, что, в частности, векторы Киллинга являются также и решениями однородного уравнения Лапласа.

        Дальнейшие выкладки приобретают громоздкий характер в случае произвольных связностей, так что пока мы ограничимся решением полученного уравнения в аффинных (декартовых) координатах, где все связности равны нулю, а метрический тензор диагонален.

        В этом случае, очевидно, ковариантные производные переходят в частные, и мы получаем уравнения на векторы Киллинга в более простой форме:
        %
        \begin{equation}
            \frac{\partial^2 \xi_c}{\partial x^a \partial x^b} = 0,
        \end{equation}
        %
        откуда, очевидно, любой вектор Киллинга должен быть представим в виде
        %
        \begin{equation}
            \xi_c = m_{ci} x^i + r_c,
        \end{equation}
        %
        причем $m_{ci} = - m_{ic}$ (следует исходного уравнения на векторы Киллинга) и не зависит от $x$, в чем можно убедиться непосредственной подстановкой.

        Дальнейшие рассуждения можно свести к следующим. Будем искать простейший вид векторов Киллинга, где все $|m_{ci}|$ и $|r_c|$ равны нулю или единице, причем для конкретного вектора либо $m_{ci} = 0$, либо $r_c = 0$. Двигаясь в этом ключе, выберем самый простой вид $r^{(i)}_c = \delta^i_c$, где $(i)$ нумерует вектор Киллинга и пробегает значения $i = 1,2,3$. Также выберем три простейших по виду ортогональных матрицы $m^{(i)}_{ci}$ так, чтобы три ее независимые компоненты были перестановками множества $\{-1, 0, 1\}$.

    \nocite{*}
    \bibliographystyle{gost705s}
    \bibliography{../../lib/doc/bib/math,../../lib/doc/bib/physics}

\end{document}
