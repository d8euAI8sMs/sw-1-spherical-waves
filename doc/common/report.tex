\documentclass[12pt,a4paper]{article}
\usepackage[T1,T2A]{fontenc}
\usepackage[utf8]{inputenc}
\usepackage[english,russian]{babel}
\usepackage{microtype}
\usepackage{csquotes}
\usepackage{amsmath}
\usepackage{amsthm}
\usepackage{amssymb}
\usepackage{mathtext}
\usepackage{physics}
\usepackage{newfloat}
\usepackage{caption}
\usepackage{indentfirst}
\usepackage{geometry}
\usepackage{hyperref}
\usepackage{mdframed}
\usepackage[inline]{enumitem}
\usepackage{graphicx}
\usepackage{subfig}
\usepackage[titletoc,toc]{appendix}

\DeclareGraphicsExtensions{.pdf,.png,.jpg,.PNG}
\graphicspath{{./img/}}
\captionsetup[figure]{justification=centering}
\renewcommand{\thesubfigure}{\asbuk{subfigure}}
\DeclareCaptionLabelSeparator{dotseparator}{. }
\captionsetup{labelsep=dotseparator}
\geometry{left=1cm,right=2cm,top=2cm,bottom=2cm}
\makeatletter\appto{\appendices}{\def\Hy@chapapp{Appendix}}\makeatother
\renewcommand{\appendixtocname}{Приложения}
\renewcommand{\appendixpagename}{Приложения}

\DeclareMathOperator{\Rot}{\mathbf{rot}}
\DeclareMathOperator{\Grad}{\mathbf{grad}}
\DeclareMathOperator{\Div}{\mathbf{div}}
\DeclareMathOperator{\D}{D}
\newcommand{\V}[1]{\mathbf{#1}}
\newcommand{\Op}[1]{\hat{\V{#1}}}

\title{Математическое приложение}
\author{Василевский А.В.}

\begin{document}

    \maketitle
    \tableofcontents

    \appendix

    %
    %
    %
    %%%%%%%%%%%%%%%%%%%%%%%%%%%%%%%%%%%%%%%%%%%%%%%%%%%%%%%%%%%%%%%%%%%%%%%
    %                           SECTION                                   %
    %%%%%%%%%%%%%%%%%%%%%%%%%%%%%%%%%%%%%%%%%%%%%%%%%%%%%%%%%%%%%%%%%%%%%%%
    %
    %
    %

    \section{Волновое уравнение}

        Получим стационарные уравнения Максвелла в свободном пространстве для монохроматической волны в нерелятивистском приближении.

        Исходная система уравнений имеет вид

        \begin{equation}\begin{aligned}\label{eq:maxwell_empty_space}
            \Rot\V{E} &= - \frac{1}{c} \frac{\partial \V{B}}{\partial t}, \\
            \Rot\V{B} &= \frac{1}{c} \frac{\partial \V{E}}{\partial t}, \\
            \Div\V{B} &= 0, \\
            \Div\V{E} &= 0.
        \end{aligned}\end{equation}
        %
        При
        %
        \begin{equation}\begin{aligned}
            \V{E}(\V{r}, t) &= \V{\hat{E}}(\V{r}) \exp(- i \omega t), \\
            \V{B}(\V{r}, t) &= \V{\hat{B}}(\V{r}) \exp(- i \omega t)
        \end{aligned}\end{equation}
        %
        уравнения принимают вид
        %
        \begin{equation}\begin{aligned}
            \Rot\V{\hat{E}} &= i \omega \frac{1}{c} \V{\hat{B}}, \\
            \Rot\V{\hat{B}} &= - i \omega \frac{1}{c} \V{\hat{E}}.
        \end{aligned}\end{equation}
        %
        Заметим, что вторые два уравнения удовлетворяются в автоматически. Взятием ротора от первого уравнения и подстановкой в него второго уравнения системы можно получить следующее выражение для вектора $\V{E}$ (аналогичное получается и для $\V{H}$):
        %
        \begin{equation}
            \Rot\Rot{\V{\hat{E}}} = i \omega \frac{1}{c} \left( -i \omega \frac{1}{c} \V{\hat{E}} \right) = \frac{\omega^2}{c^2} \V{\hat{E}}.
        \end{equation}
        %
        Полученное уравнение называется стационарным волновым уравнением.

        Его можно упростить, принимая во внимание вторую пару уравнений \autoref{eq:maxwell_empty_space} и тождество
        %
        \begin{equation}\label{eq:laplacian_vect}
            \Rot\Rot = \Grad\Div - \Delta,
        \end{equation}
        %
        где $\Delta$ -- векторный оператор Лапласа:
        %
        \begin{equation}
            \Delta\V{\hat{E}} = - \frac{\omega^2}{c^2} \V{\hat{E}}.
        \end{equation}

        В дальнейшем условимся опускать знак \enquote{$\hat{\ }$}, имея в виду именно не зависящую от времени часть $\V{E}$, если не сказано обратного.

    %
    %
    %
    %%%%%%%%%%%%%%%%%%%%%%%%%%%%%%%%%%%%%%%%%%%%%%%%%%%%%%%%%%%%%%%%%%%%%%%
    %                           SECTION                                   %
    %%%%%%%%%%%%%%%%%%%%%%%%%%%%%%%%%%%%%%%%%%%%%%%%%%%%%%%%%%%%%%%%%%%%%%%
    %
    %
    %

    \section{Векторный оператор Лапласа в произвольных координатах}

        Ограничимся трехмерным евклидовым пространством. Получим явный инвариантный вид введенного оператора Лапласа. Будем опираться на формулировку \autoref{eq:laplacian_vect}.

        Цель данного пункта -- показать, что введенный выше векторный оператор Лапласа не является особым объектом, принадлежит к классу дифференциальных операторов второго порядка, а его вид в координатной записи не зависит от поля, к которому он применяется.

        Ротор в криволинейных координатах через ковариантные производные от ковариантных компонент вектора $\V{a}$ представлен в виде
        %
        \begin{equation}
            \left( \Rot\V{a} \right)^i
                = \frac{1}{2} \varepsilon^{ijk} \left(
                    \nabla_k a_j - \nabla_j a_k
                \right)
                \equiv \varepsilon^{ijk} \nabla_{[k} a_{j]}.
        \end{equation}
        %
        Распишем $\Rot\Rot\V{a}$:
        %
        \begin{equation}
            \left( \Rot\Rot\V{a} \right)^i
                = \varepsilon^{ijk} \nabla_{[k} \left( \Rot\V{a} \right)_{j]}.
        \end{equation}
        %
        Получим явный вид альтернированной ковариантной производной ротора:
        %
        \begin{equation}\begin{aligned}
            \nabla_{[k} \left( \Rot\V{a} \right)_{j]}
                &= \nabla_{[k} \left( g_{j]m} \left( \Rot\V{a} \right)^m \right) \\
                &= g_{[jm} \nabla_{k]} \left( \Rot\V{a} \right)^m \\
                &= g_{[jm} \nabla_{k]} \left(
                       \varepsilon^{mqr} \nabla_{[r} a_{q]}
                \right) \\
                &= g_{[jm} \varepsilon^{mqr} \left(
                   \nabla_{k]} \nabla_{[r} a_{q]}
                \right),
        \end{aligned}\end{equation}
        %
        откуда
        %
        \begin{equation}
            \left( \Rot\Rot\V{a} \right)^i
                = \varepsilon^{ijk} g_{[jm} \varepsilon^{mqr} \left(
                    \nabla_{k]} \nabla_{[r} a_{q]}
                \right).
        \end{equation}
        %
        Распишем произведение перед скобкой, пользуясь тождеством
        %
        \begin{equation}
           \varepsilon_{ijk} \varepsilon^{imn} = \delta_j^m \delta_k^n - \delta_j^n \delta_k^m,
        \end{equation}
        %
        получим
        %
        \begin{equation}\begin{aligned}
           \varepsilon^{ijk} g_{jm} \varepsilon^{mqr}
                &= \varepsilon^{kij} g_{jm} \varepsilon^{mqr} \\
                &= g^{ip} g^{ks} \varepsilon_{msp} \varepsilon^{mqr} \\
                &= g^{ip} g^{ks} \left(
                    \delta_s^q \delta_p^r - \delta_s^r \delta_p^q
                \right)
        \end{aligned}\end{equation}
        %
        и аналогично для альтернированного по индексам $[jk]$ варианта
        %
        \begin{equation}\begin{aligned}
           \varepsilon^{ijk} g_{km} \varepsilon^{mqr}
                &= \varepsilon^{ijk} g_{jm} \varepsilon^{mqr} \\
                &= g^{ip} g^{js} \varepsilon_{mps} \varepsilon^{mqr} \\
                &= g^{ip} g^{js} \left(
                    \delta_p^q \delta_s^r - \delta_p^r \delta_s^q
                \right) \\
                &= - g^{ip} g^{js} \left(
                    \delta_s^q \delta_p^r - \delta_s^r \delta_p^q
                \right).
        \end{aligned}\end{equation}

        В итоге, в явном виде ротор ротора перепишется следующим образом:
        %
        \begin{equation}\begin{aligned}
            \left( \Rot\Rot\V{a} \right)^i
                &= \frac{1}{2} g^{ip} \left\{
                       g^{ks} \left(
                           \delta_s^q \delta_p^r - \delta_s^r \delta_p^q
                       \right) \nabla_k \nabla_{[r} a_{q]} +
                       g^{js} \left(
                           \delta_s^q \delta_p^r - \delta_s^r \delta_p^q
                       \right) \nabla_j \nabla_{[r} a_{q]}
                   \right\} \\
                &= g^{ip} \left(
                       g^{kq} \nabla_k \nabla_{[p} a_{q]} - g^{kr} \nabla_k \nabla_{[r} a_{p]}
                   \right) \\
                &= 2 g^{ip} g^{kq} \left(
                       \nabla_k \nabla_{[p} a_{q]}
                   \right) \\
                &= g^{ip} g^{kq} \left(
                       \nabla_k \nabla_{p} a_{q} - \nabla_k \nabla_{q} a_{p}
                   \right) \\
                &= g^{ip} \nabla_k \nabla_{p} a^k - g^{kq} \nabla_k \nabla_{q} a^i
        \end{aligned}\end{equation}
        %
        Пользуясь теперь
        %
        \begin{equation}\begin{gathered}
            (\nabla_k \nabla_p - \nabla_p \nabla_k) a^i = - R_{kpm}^{\hphantom{kpm}i} a^m , \\
            R_{kpm}^{\hphantom{kpm}i} \delta_i^k = R_{kpm}^{\hphantom{kpm}k} = R_{pm} ,
        \end{gathered}\end{equation}
        %
        где $R_{kpm}^{\hphantom{kpm}i}$ -- тензор кривизны (тензор Римана-Кристоффеля), а $R_{pm}$, соответственно, тензор Риччи, преобразуем последнюю строчку к виду
        %
        \begin{equation}\begin{aligned}
            \left( \Rot\Rot\V{a} \right)^i
                &= g^{ip} \nabla_k \nabla_p a^k - g^{kq} \nabla_k \nabla_q a^i \\
                &= g^{ip} \nabla_p \nabla_k a^k
                    + g^{ip} R_{pm} a^m
                    - g^{kq} \nabla_k \nabla_q a^i \\
                &= \left( \Grad\Div{\V{a}} - \Delta{\V{a}} \right)^i + g^{ip} R_{pm} a^m
        \end{aligned}\end{equation}
        %
        В пространствах с нулевой кривизной (а мы рассматриваем такие пространства) тензор Риччи тождественно равен нулю.

        Итак, мы получили общий вид оператора Лапласа в произвольной системе координат. Выпишем его отдельно:
        %
        \begin{equation}
            \Delta{\V{a}} = g^{kq} \nabla_k \nabla_{q} a^i
                          = \nabla^k \nabla_k a^i
        \end{equation}
        %
        Мы также видим, что вид оператора Лапласа не зависит от объекта, к которому он применяется (скалярное, векторное, тензорное поле). Оператор Лапласа является частным случаем более общего семейства линейных дифференциальных операторов порядка $p$, порождаемых определенным тензорным полем:
        %
        \begin{equation}
            \D^p\tau = a^{i_1 \dots i_p} \nabla_{i_1} \dots \nabla_{i_p} \tau ,
        \end{equation}
        %
        где под $\tau$ понимается тензорное поле произвольной (в т.ч. нулевой) валентности. Ниже будут рассмотрены коммутационные соотношения и условия коммутации между операторами вплоть до операторов второго порядка.

    %
    %
    %
    %%%%%%%%%%%%%%%%%%%%%%%%%%%%%%%%%%%%%%%%%%%%%%%%%%%%%%%%%%%%%%%%%%%%%%%
    %                           SECTION                                   %
    %%%%%%%%%%%%%%%%%%%%%%%%%%%%%%%%%%%%%%%%%%%%%%%%%%%%%%%%%%%%%%%%%%%%%%%
    %
    %
    %

    \section{Ли-вариация, поля, операторы Киллинга}

        Будем рассматривать, как обычно, трехмерные пространства без кручения.

        Ли-вариация является инвариантным обобщением производной по направлению на произвольные тензорные поля. Для произвольного тензорного поля $\tau(P)$ вводится оператор
        %
        \begin{equation}
            \delta_\xi \tau(P) = \lim\limits_{\varepsilon \to 0} \frac{
                \tau(P + \varepsilon \xi) - \tau(P)
            }{\varepsilon}.
        \end{equation}
        %

        Второе эквивалентное определение (\cite{lie_derivative_theory}) состоит в следующем. Рассмотрим тензорное поле $\tau(\xi)$ и бесконечно малое координатное преобразование $T: \xi \rightarrow {'\xi}$ такое, что в заданной некоторой системе $\varkappa$ $T$ выражается преобразованием $'\xi^i = \xi^i + v^i d\varepsilon$, где $v^i$ -- некоторое векторное поле. При применении преобразования координат $T^{-1}: \varkappa \rightarrow {'\varkappa}$, причем $'\xi^i \rightarrow \xi^{i'}$, мы получим тензор $'\tau^{'\varkappa}(\xi)$, заданный, однако, в $'\varkappa$. Наконец, рассмотрим тензор $'\tau(\xi)$, заданный в $\varkappa$ и имеющий те же координаты, что и полученный $'\tau^{'\varkappa}(\xi)$ в $'\varkappa$. Ли-вариацией тензорного поля $\tau(\xi)$ по векторному полю $v$ называется разность между $\tau(\xi)$ и $'\tau(\xi)$, деленная на $d\varepsilon$.

        Ли-вариация тензорного поля ${T^{a_1 \dots a_p}}_{b_1 \dots b_q}(M)$ по направлению векторного поля $\xi(M)$ в координатной записи имеет вид:
        %
        \begin{equation}\begin{aligned}
            \delta_\xi {T^{a_1 \dots a_p}}_{b_1 \dots b_q}
                &= \xi^c \left( \nabla_c {T^{a_i \dots a_p}}_{b_1 \dots b_q} \right) \\
                &- \left( \nabla_{c} \xi^{a_1} \right) {T^{c a_2 \dots a_p}}_{b_1 \dots b_q} - \dots
                 - \left( \nabla_{c} \xi^{a_p} \right) {T^{a_1 \dots a_{p-1} c}}_{b_1 \dots b_q} \\
                &+ \left( \nabla_{b_1} \xi^c \right) {T^{a_1 \dots a_p}}_{c b_2 \dots b_q} + \dots
                 + \left( \nabla_{b_q} \xi^c \right) {T^{a_1 \dots a_p}}_{b_1 \dots b_{q-1} c}
        \end{aligned}\end{equation}
        %
        В частности, для векторных полей справедливо
        %
        \begin{equation}\begin{aligned}\label{eq:vector_field_commutator}
            \delta_\xi T^a
                = \xi^c \nabla_c T^a - T^c \nabla_{c} \xi^a
                \equiv [\xi, T]^a.
        \end{aligned}\end{equation}
        %
        Этим вводится понятие коммутатора двух векторных полей (Ли-коммутатор, скобка Ли).

        Векторные поля, на которых Ли-вариация метрического тензора равна нулю, называются движениями, или полями Киллинга:
        %
        \begin{equation}\begin{aligned}
            \delta_\xi g_{ab}
                &= \xi^c \nabla_c g_{ab} + g_{cb} \nabla_{c} \xi^a + g_{ac} \nabla_{c} \xi^a \\
                &= g_{cb} \nabla_{a} \xi^c + g_{ac} \nabla_{b} \xi^c \\
                &= \nabla_{a} \xi_b + \nabla_{b} \xi_a \\
                &= 2 \nabla_{(a} \xi_{b)}.
        \end{aligned}\end{equation}

        Найдем векторы Киллинга трехмерного евклидова пространства. Запишем условие инвариантности метрики:
        %
        \begin{equation}
            \delta_\xi g_{ab}
                = 2 \nabla_{(a} \xi_{b)}
                = 0
                \Leftrightarrow \nabla_{(a} \xi_{b)} = 0.
        \end{equation}
        %
        Продифференцируем полученное выражение
        %
        \begin{equation}
            \nabla_c \nabla_{(a} \xi_{b)} = 0,
        \end{equation}
        %
        трижды проведем циклирование по индексам
        %
        \begin{equation}\begin{aligned}
            \nabla_c \nabla_{(a} \xi_{b)} &= 0 \\
            \nabla_a \nabla_{(b} \xi_{c)} &= 0 \\
            \nabla_b \nabla_{(c} \xi_{a)} &= 0
        \end{aligned}\end{equation}
        %
        Сложим первое и второе уравнения, третье вычтем. Получим
        %
        \begin{equation}\begin{aligned}
            \nabla_c \nabla_{(a} \xi_{b)} +
            \nabla_a \nabla_{(b} \xi_{c)} -
            \nabla_b \nabla_{(c} \xi_{a)} =
            2 \nabla_a \nabla_b \xi_c     = 0,
        \end{aligned}\end{equation}
        %
        откуда получаем систему уравнений на векторы Киллинга:
        %
        \begin{equation}
            \nabla_a \nabla_b \xi_c = 0.
        \end{equation}
        %
        Полученное выражение очень похоже на требование равенства нулю оператора Лапласа $\Delta \equiv g^{ab} \nabla_a \nabla_b$, но отличается от него отсутствием свертки с метрическим тензором, что \enquote{поправимо} путем искусственной свертки данного уравнения с метрическим тензором. Отсюда видно, что, в частности, векторы Киллинга являются также и решениями однородного уравнения Лапласа.

        Дальнейшие выкладки приобретают громоздкий характер в случае произвольных связностей, так что пока мы ограничимся решением полученного уравнения в аффинных (декартовых) координатах, где все связности равны нулю, а метрический тензор диагонален.

        В этом случае, очевидно, ковариантные производные переходят в частные, и мы получаем уравнения на векторы Киллинга в более простой форме:
        %
        \begin{equation}
            \frac{\partial^2 \xi_c}{\partial x^a \partial x^b} = 0,
        \end{equation}
        %
        откуда, очевидно, любой вектор Киллинга должен быть представим в виде
        %
        \begin{equation}
            \xi_c = m_{ci} x^i + r_c,
        \end{equation}
        %
        причем $m_{ci} = - m_{ic}$ (следует исходного уравнения на векторы Киллинга) и не зависит от $x$, в чем можно убедиться непосредственной подстановкой.

        Дальнейшие рассуждения можно свести к следующим. Будем искать простейший вид векторов Киллинга, где все $|m_{ci}|$ и $|r_c|$ равны нулю или единице, причем для конкретного вектора либо $m_{ci} = 0$, либо $r_c = 0$. Двигаясь в этом ключе, выберем самый простой вид $r^{(i)}_c = \delta^i_c$, где $(i)$ нумерует вектор Киллинга и пробегает значения $i = 1,2,3$. Также выберем три простейших по виду ортогональных матрицы $m^{(i)}_{cj} = \varepsilon^{\{i}_{jk\}}$, где за $\{ijk\}$ обозначена одна из трех возможных нечетных перестановок индексов $i,j,k$.

        Выпишем явный вид всех полученных векторов Киллинга:
        %
        \begin{equation}
            \{ \xi^{(i)}_j \}
            =
            \left[
                \V{n}_x, \V{n}_y, \V{n}_z,
                \V{l}_x, \V{l}_y, \V{l}_z
            \right]
            =
            \begin{bmatrix}
                1 & 0 & 0 & 0  & -z & y  \\
                0 & 1 & 0 & z  & 0  & -x \\
                0 & 0 & 1 & -y & x  & 0
            \end{bmatrix}
        \end{equation}

        Из инвариантности определения вектора (тензора) следует, что и в любой другой системе координат полученные векторы будет являться полями Киллинга.

    %
    %
    %
    %%%%%%%%%%%%%%%%%%%%%%%%%%%%%%%%%%%%%%%%%%%%%%%%%%%%%%%%%%%%%%%%%%%%%%%
    %                           SECTION                                   %
    %%%%%%%%%%%%%%%%%%%%%%%%%%%%%%%%%%%%%%%%%%%%%%%%%%%%%%%%%%%%%%%%%%%%%%%
    %
    %
    %

    \section{Дифференциальные операторы и их коммутаторы\label{sec:commutators}}

        Линейным дифференциальным оператором называется конструкция вида:
        %
        \begin{equation}
            \D_a\tau = a^{i_1 \dots i_p} \nabla_{i_1} \dots \nabla_{i_p} \tau ,
        \end{equation}
        %
        где $\tau$ -- некоторое дифференцируемое тензорное поле. $a^{i_1 \dots i_p}$ называется порождающим тензорным полем дифференциального оператора.

        Важным является вопрос коммутации дифференциальных операторов. Два оператора коммутируют, если их коммутатор равен нулю:
        %
        \begin{equation}
            [\D_a, \D_b]\tau = (\D_a \D_b - \D_b \D_a) \tau = 0 .
        \end{equation}
        %
        В противном случае говорят, что операторы не коммутируют.

        Коммутация операторов влечет простые следствия -- у коммутирующих операторов собственные функции являются общими, что позволяет разбить задачу поиска собственных функций оператора высокого порядка на ряд задач поиска собственных функций операторов меньших порядков.

        Рассмотрим два важных случая.

        \subsection{Коммутация оператора Лапласа с дифференциальными операторами первого порядка}

            Выясним условия коммутации некоторого дифференциального оператора первого порядка $D_\xi = \xi^i \nabla_i$ и оператора Лапласа $\Delta = D_g = g^{ij} \nabla_i \nabla_j$. Можно показать (\cite{differential_operator_commutators}), что для $[\Delta, D_\xi] = 0$ необходимо, чтобы векторное поле $\xi$ являлось полем Киллинга, в чем можно убедиться непосредственной подстановкой.

        \subsection{Коммутация операторов первого порядка между собой}

            Для дифференциальных операторов первого порядка получаем:
            %
            \begin{equation}\begin{aligned}
                \left[ \D_\xi, \D_\eta \right] \tau
                    &= (\xi^i \nabla_i \eta^j - \eta^i \nabla_i \xi^j) \circ \nabla_j \tau \\
                    &= (\xi^i \nabla_i \eta^j - \eta^i \nabla_i \xi^j) \nabla_j \tau
                        + \xi^i \eta^j (\nabla_i \nabla_j - \nabla_j \nabla_i) \tau \\
                    &= \left[ \xi, \eta \right]^j \nabla_j \tau
                        + \xi^i \eta^j (\nabla_i \nabla_j - \nabla_j \nabla_i) \tau
            \end{aligned}\end{equation}
            %
            С помощью \enquote{$\circ$} явно подчеркнут операторный характер выражения в скобках. Коммутатор векторных полей понимается в смысле \autoref{eq:vector_field_commutator}.

            При наличии кривизны пространства ($\nabla_i \nabla_j - \nabla_j \nabla_i \neq 0$) условие их коммутации приобретает сложный характер. В неискривленном пространстве операторы коммутируют, если коммутируют порождающие их векторные поля. Из этого также следует, что коммутатор двух дифференциальных операторов первого порядка является оператором, порожденным коммутатором порождающих их векторных полей. Он также является дифференциальным оператором первого порядка.

            Дифференциальны операторы, порожденные полями Киллинга, называются операторами Киллинга.

            Для удобства операторы первого порядка часто обозначаются символом \enquote{$\hat{ }$} над порождающим векторным полем: $D_\xi \equiv \Op\xi$.

    %
    %
    %
    %%%%%%%%%%%%%%%%%%%%%%%%%%%%%%%%%%%%%%%%%%%%%%%%%%%%%%%%%%%%%%%%%%%%%%%
    %                           SECTION                                   %
    %%%%%%%%%%%%%%%%%%%%%%%%%%%%%%%%%%%%%%%%%%%%%%%%%%%%%%%%%%%%%%%%%%%%%%%
    %
    %
    %

    \section{Коммутационные соотношения операторов Киллинга}

        Найдем коммутационные соотношения внутри и между группами движений евклидова пространства:
        %
        \begin{equation}
            \begin{bmatrix}
                [ \Op{n}_x \Op{n}_y ] & [ \Op{n}_z \Op{n}_x ] & [ \Op{n}_y \Op{n}_z ] \\
                [ \Op{l}_x \Op{l}_y ] & [ \Op{l}_z \Op{l}_x ] & [ \Op{l}_y \Op{l}_z ] \\
                [ \Op{n}_x \Op{l}_x ] & [ \Op{n}_y \Op{l}_x ] & [ \Op{n}_z \Op{l}_x ] \\
                [ \Op{n}_x \Op{l}_y ] & [ \Op{n}_y \Op{l}_y ] & [ \Op{n}_z \Op{l}_y ] \\
                [ \Op{n}_x \Op{l}_z ] & [ \Op{n}_y \Op{l}_z ] & [ \Op{n}_z \Op{l}_z ]
            \end{bmatrix}
            =
            \begin{bmatrix}
                0          &   0        &   0        \\
                  \Op{l}_z &   \Op{l}_y &   \Op{l}_x \\
                0          &   \Op{n}_z & - \Op{n}_y \\
                - \Op{n}_z &   0        &   \Op{n}_x \\
                  \Op{n}_y & - \Op{n}_x &   0
            \end{bmatrix}
        \end{equation}
        %

        Полученные коммутаторы инвариантны относительно выбора системы координат.

        Из \autoref{sec:commutators} видно, что операторы сдвигов и вращений коммутируют с оператором Лапласа.

        Составим оператор
        %
        \begin{equation}
            \Op{l}^2 = \Op{l}_x \Op{l}_x + \Op{l}_y \Op{l}_y + \Op{l}_z \Op{l}_z .
        \end{equation}
        %
        Очевидно, он коммутирует с любыми из операторов вращений $\Op{l}_i$, в чем можно убедиться непосредственной проверкой. Также он коммутирует с оператором Лапласа (применено правило Лейбница и тот факт, что оператор Лапласа коммутирует с любыми из операторов поворота):
        %
        \begin{equation}\begin{aligned}
            \left[ \Delta, \Op{l}^2 \right]
                &= \left[ \Delta, \Op{l}_x \Op{l}_x \right]
                    + \left[ \Delta, \Op{l}_y \Op{l}_y \right]
                    + \left[ \Delta, \Op{l}_z \Op{l}_z \right] \\
                &= \left[ \Delta, \Op{l}_x \right] \Op{l}_x
                    + \Op{l}_x \left[ \Delta, \Op{l}_x \right]
                    + \left[ \Delta, \Op{l}_y \right] \Op{l}_y
                    + \Op{l}_y \left[ \Delta, \Op{l}_y \right]
                    + \left[ \Delta, \Op{l}_z \right] \Op{l}_z
                    + \Op{l}_z \left[ \Delta, \Op{l}_z \right] \\
                &= 0
        \end{aligned}\end{equation}

    %
    %
    %
    %%%%%%%%%%%%%%%%%%%%%%%%%%%%%%%%%%%%%%%%%%%%%%%%%%%%%%%%%%%%%%%%%%%%%%%
    %                           SECTION                                   %
    %%%%%%%%%%%%%%%%%%%%%%%%%%%%%%%%%%%%%%%%%%%%%%%%%%%%%%%%%%%%%%%%%%%%%%%
    %
    %
    %

    \section{Поля Киллинга в сферической системе координат}

        Из инвариантности объекта \enquote{вектор}, задающего направление Ли-вариации, следует, что поля Киллинга, найденные в одной системе координат, останутся полями Киллинга и в другой системе координат. Найдем векторы Киллинга в сферической координатной системе с метрикой
        %
        \begin{equation}
            dl^2 = dr^2 + r^2 (d\theta^2 + \sin ^2 \theta d\varphi^2)
        \end{equation}
        %
        и связью между координатами декартовой и сферической системами координат
        %
        \begin{equation}
            \V{x}(r, \theta, \varphi)
            =
            \{x^i\}
            =
            \begin{bmatrix}
                x \\ y \\ z
            \end{bmatrix}
            =
            \begin{bmatrix}
                r \sin\theta \cos\varphi \\
                r \sin\theta \sin\varphi \\
                r \cos\theta
            \end{bmatrix}
            .
        \end{equation}
        %
        Для чего перейдем к новой системе координат:
        %
        \begin{equation}\begin{aligned}
            \xi_{i'} = \frac{\partial x^j}{\partial x^{i'}} \xi_j.
        \end{aligned}\end{equation}
        %
        Проведенные вычисления дают
        %
        \begin{equation}
            \{\V{n}^i_j\}
            =
            \begin{bmatrix}
                \sin\theta \cos\varphi     & \sin\theta \sin\varphi   & \cos\theta \\
                r \cos\theta \cos\varphi   & r \cos\theta \sin\varphi & - r \sin\theta \\
                - r \sin\theta \sin\varphi & r \sin\theta \cos\varphi & 0
            \end{bmatrix}
            ,
        \end{equation}
        %
        \begin{equation}
            \{\V{l}^i_j\}
            =
            \begin{bmatrix}
                0
                    & 0
                    & 0 \\
                r^2 \sin\varphi
                    & - r^2 \cos\varphi
                    & 0 \\
                r^2 \cos\varphi \cos\theta \sin\theta
                    & r^2 \sin\varphi \cos\theta \sin\theta
                    & - r^2 \sin^2\theta
            \end{bmatrix}
            .
        \end{equation}

    %
    %
    %
    %%%%%%%%%%%%%%%%%%%%%%%%%%%%%%%%%%%%%%%%%%%%%%%%%%%%%%%%%%%%%%%%%%%%%%%
    %                           SECTION                                   %
    %%%%%%%%%%%%%%%%%%%%%%%%%%%%%%%%%%%%%%%%%%%%%%%%%%%%%%%%%%%%%%%%%%%%%%%
    %
    %
    %

    \section{Собственные функции и собственные значения операторов вращений Киллинга}

        Найдем связь между собственными значениями и собственными функциями дифференциальных операторов первого порядка, построенных на полях Киллинга, а конкретно на вращениях. Будем называть собственные функции модами, причем природа этих функций (скалярное, векторное, тензорное поле) нас не интересует.

        Введем операторы $\Op{l}_{+} = \Op{l}_x - i\Op{l}_y$ и $\Op{l}_{-} = \Op{l}_x + i\Op{l}_y$. Коммутационные соотношения между ними и оператором $\Op{l}_z$:
        %
        \begin{equation}
            \left[ \Op{l}_{+}, \Op{l}_{-} \right] = 2 i \Op{l}_{z} ;
            \left[ \Op{l}_{z}, \Op{l}_{+} \right] =   i \Op{l}_{+} ;
            \left[ \Op{l}_{z}, \Op{l}_{-} \right] = - i \Op{l}_{-} .
        \end{equation}
        %

        Пусть мода $h_m$ является собственной функцией оператора $\Op{l}_z$ с собственным значением $i m$. Тогда мода $h_{+} = \Op{l}_{+} h_m$ тоже будет являться собственной функцией оператора $\Op{l}_z$ с собственным значением $i (m + 1)$: $h_{+} = h_{m + 1}$. Аналогично, $h_{-} = h_{m - 1}$. В этом можно убедиться непосредственной проверкой. В силу этого $\Op{l}_{+}$ и $\Op{l}_{-}$ называют операторами повышения и понижения.

        Пусть теперь мода $g_\lambda$ является является собственной функцией оператора $\Op{l}^2$ с собственным значением $\lambda$. Поскольку оператор $\Op{l}^2$ коммутирует с $\Op{l}_z$, мода $g_\lambda$ также является и собственной функцией оператора $\Op{l}_z$, а значит можно обозначить $g_\lambda = h_m = h_{\lambda,m}$.

        Найдем связь между $\lambda$ и $m$. Максимальное значение $m$, т.е. значение, при котором $\Op{l}_{+} h_{\lambda,m} = 0$, обозначим $l$. Представим оператор $\Op{l}^2$ в виде
        %
        \begin{equation}\begin{aligned}
            \Op{l}^2
                &= \Op{l}_{-}\Op{l}_{+} + \Op{l}^2_{z} + i \Op{l}_{z} \\
                &= \Op{l}_{+}\Op{l}_{-} + \Op{l}^2_{z} - i \Op{l}_{z} .
        \end{aligned}\end{equation}
        %
        Подействуем оператором $\Op{l}_{-}\Op{l}_{+}$ на моду $h_{\lambda,l}$:
        %
        \begin{equation}\begin{aligned}
            \Op{l}_{-}\Op{l}_{+} h_{\lambda,l}
                &= (\Op{l}^2 - \Op{l}^2_{z} - i \Op{l}_{z}) h_{\lambda,l} \\
                &= (\lambda + l^2 + l) h_{\lambda,l} \\
                &= 0 ,
        \end{aligned}\end{equation}
        %
        откуда следует, что $\lambda = - l (l + 1)$. Применяя другой оператор, $\Op{l}_{+}\Op{l}_{-}$, мы увидим, что минимальное значение $m = - l$.

    %
    %
    %
    %%%%%%%%%%%%%%%%%%%%%%%%%%%%%%%%%%%%%%%%%%%%%%%%%%%%%%%%%%%%%%%%%%%%%%%
    %                        BIBLIOGRAPHY                                 %
    %%%%%%%%%%%%%%%%%%%%%%%%%%%%%%%%%%%%%%%%%%%%%%%%%%%%%%%%%%%%%%%%%%%%%%%
    %
    %
    %

    \nocite{*}
    \bibliographystyle{../../lib/doc/bib/utf8gost705s}
    \bibliography{../../lib/doc/bib/math,../../lib/doc/bib/physics}

\end{document}
