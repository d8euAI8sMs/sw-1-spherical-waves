\documentclass[12pt,a4paper]{article}
\usepackage[T1,T2A]{fontenc}
\usepackage[utf8]{inputenc}
\usepackage[english,russian]{babel}
\usepackage{microtype}
\usepackage{csquotes}
\usepackage{amsmath}
\usepackage{amsthm}
\usepackage{amssymb}
\usepackage{mathtext}
\usepackage{newfloat}
\usepackage{caption}
\usepackage{indentfirst}
\usepackage{geometry}
\usepackage{hyperref}
\usepackage{mdframed}
\usepackage[inline]{enumitem}
\usepackage{graphicx}
\usepackage{subfig}
\usepackage[titletoc,toc]{appendix}

\DeclareGraphicsExtensions{.pdf,.png,.jpg,.PNG}
\graphicspath{{./img/}}
\captionsetup[figure]{justification=centering}
\renewcommand{\thesubfigure}{\asbuk{subfigure}}
\DeclareCaptionLabelSeparator{dotseparator}{. }
\captionsetup{labelsep=dotseparator}
\geometry{left=1cm,right=2cm,top=2cm,bottom=2cm}
\makeatletter\appto{\appendices}{\def\Hy@chapapp{Appendix}}\makeatother
\renewcommand{\appendixtocname}{Приложения}
\renewcommand{\appendixpagename}{Приложения}

\DeclareMathOperator{\Rot}{\mathbf{rot}}
\DeclareMathOperator{\Grad}{\mathbf{grad}}
\DeclareMathOperator{\Div}{\mathbf{div}}
\newcommand{\V}[1]{\mathbf{#1}}

\title{Математическое приложение}
\author{Василевский А.В.}

\begin{document}

    \maketitle
    \tableofcontents

    \appendix

    %
    %
    %
    %%%%%%%%%%%%%%%%%%%%%%%%%%%%%%%%%%%%%%%%%%%%%%%%%%%%%%%%%%%%%%%%%%%%%%%
    %                           SECTION                                   %
    %%%%%%%%%%%%%%%%%%%%%%%%%%%%%%%%%%%%%%%%%%%%%%%%%%%%%%%%%%%%%%%%%%%%%%%
    %
    %
    %

    \section{Волновое уравнение}

        Получим стационарные уравнения Максвелла в свободном пространстве для монохроматической волны в нерелятивистском приближении.

        Исходная система уравнений имеет вид

        \begin{equation}\begin{aligned}\label{eq:maxwell_empty_space}
            \Rot\V{E} &= - \frac{1}{c} \frac{\partial \V{B}}{\partial t}, \\
            \Rot\V{B} &= \frac{1}{c} \frac{\partial \V{E}}{\partial t}, \\
            \Div\V{B} &= 0, \\
            \Div\V{E} &= 0.
        \end{aligned}\end{equation}
        %
        При
        %
        \begin{equation}\begin{aligned}
            \V{E}(\V{r}, t) &= \V{\hat{E}}(\V{r}) \exp(- i \omega t), \\
            \V{B}(\V{r}, t) &= \V{\hat{B}}(\V{r}) \exp(- i \omega t)
        \end{aligned}\end{equation}
        %
        уравнения принимают вид
        %
        \begin{equation}\begin{aligned}
            \Rot\V{\hat{E}} &= i \omega \frac{1}{c} \V{\hat{B}}, \\
            \Rot\V{\hat{B}} &= - i \omega \frac{1}{c} \V{\hat{E}}.
        \end{aligned}\end{equation}
        %
        Заметим, что вторые два уравнения удовлетворяются в автоматически. Взятием ротора от первого уравнения и подстановкой в него второго уравнения системы можно получить следующее выражение для вектора $\V{E}$ (аналогичное получается и для $\V{H}$):
        %
        \begin{equation}
            \Rot\Rot{\V{\hat{E}}} = i \omega \frac{1}{c} \left( -i \omega \frac{1}{c} \V{\hat{E}} \right) = \frac{\omega^2}{c^2} \V{\hat{E}}.
        \end{equation}
        %
        Полученное уравнение называется стационарным волновым уравнением.

        Его можно упростить, принимая во внимание вторую пару уравнений \autoref{eq:maxwell_empty_space} и тождество
        %
        \begin{equation}\label{eq:laplacian_vect}
            \Rot\Rot = \Grad\Div - \Delta,
        \end{equation}
        %
        где $\Delta$ -- векторный оператор Лапласа:
        %
        \begin{equation}
            \Delta\V{\hat{E}} = - \frac{\omega^2}{c^2} \V{\hat{E}}.
        \end{equation}

        В дальнейшем условимся опускать знак \enquote{$\hat{\ }$}, имея в виду именно не зависящую от времени часть $\V{E}$, если не сказано обратного.

    %
    %
    %
    %%%%%%%%%%%%%%%%%%%%%%%%%%%%%%%%%%%%%%%%%%%%%%%%%%%%%%%%%%%%%%%%%%%%%%%
    %                           SECTION                                   %
    %%%%%%%%%%%%%%%%%%%%%%%%%%%%%%%%%%%%%%%%%%%%%%%%%%%%%%%%%%%%%%%%%%%%%%%
    %
    %
    %

    \section{Векторный оператор Лапласа в произвольных координатах}

        Ограничимся трехмерным евклидовым пространством. Получим явный инвариантный вид введенного оператора Лапласа. Будем опираться на формулировку \autoref{eq:laplacian_vect}.

        Ротор в криволинейных координатах через ковариантные производные от ковариантных компонент вектора $\V{a}$ представлен в виде
        %
        \begin{equation}
            \left( \Rot\V{a} \right)^i
                = \frac{1}{2} \varepsilon^{ijk} \left(
                    \nabla_k a_j - \nabla_j a_k
                \right)
                \equiv \varepsilon^{ijk} \nabla_{[k} a_{j]}.
        \end{equation}
        %
        Распишем $\Rot\Rot\V{a}$:
        %
        \begin{equation}
            \left( \Rot\Rot\V{a} \right)^i
                = \varepsilon^{ijk} \nabla_{[k} \left( \Rot\V{a} \right)_{j]}.
        \end{equation}
        %
        Получим явный вид альтернированной ковариантной производной ротора:
        %
        \begin{equation}\begin{aligned}
            \nabla_{[k} \left( \Rot\V{a} \right)_{j]}
                &= \nabla_{[k} \left( g_{j]m} \left( \Rot\V{a} \right)^m \right) \\
                &= g_{[jm} \nabla_{k]} \left( \Rot\V{a} \right)^m \\
                &= g_{[jm} \nabla_{k]} \left(
                       \varepsilon^{mqr} \nabla_{[r} a_{q]}
                \right) \\
                &= g_{[jm} \varepsilon^{mqr} \left(
                   \nabla_{k]} \nabla_{[r} a_{q]}
                \right),
        \end{aligned}\end{equation}
        %
        откуда
        %
        \begin{equation}
            \left( \Rot\Rot\V{a} \right)^i
                = \varepsilon^{ijk} g_{[jm} \varepsilon^{mqr} \left(
                    \nabla_{k]} \nabla_{[r} a_{q]}
                \right).
        \end{equation}
        %
        Распишем произведение перед скобкой, пользуясь тождеством
        %
        \begin{equation}
           \varepsilon_{ijk} \varepsilon^{imn} = \delta_j^m \delta_k^n - \delta_j^n \delta_k^m,
        \end{equation}
        %
        получим
        %
        \begin{equation}\begin{aligned}
           \varepsilon^{ijk} g_{jm} \varepsilon^{mqr}
                &= \varepsilon^{kij} g_{jm} \varepsilon^{mqr} \\
                &= g^{ip} g^{ks} \varepsilon_{msp} \varepsilon^{mqr} \\
                &= g^{ip} g^{ks} \left(
                    \delta_s^q \delta_p^r - \delta_s^r \delta_p^q
                \right)
        \end{aligned}\end{equation}
        %
        и аналогично для альтернированного по индексам $[jk]$ варианта
        %
        \begin{equation}\begin{aligned}
           \varepsilon^{ijk} g_{km} \varepsilon^{mqr}
                &= \varepsilon^{ijk} g_{jm} \varepsilon^{mqr} \\
                &= g^{ip} g^{js} \varepsilon_{mps} \varepsilon^{mqr} \\
                &= g^{ip} g^{js} \left(
                    \delta_p^q \delta_s^r - \delta_p^r \delta_s^q
                \right) \\
                &= - g^{ip} g^{js} \left(
                    \delta_s^q \delta_p^r - \delta_s^r \delta_p^q
                \right).
        \end{aligned}\end{equation}

        В итоге, в явном виде ротор ротора перепишется следующим образом:
        %
        \begin{equation}\begin{aligned}
            \left( \Rot\Rot\V{a} \right)^i
                &= \frac{1}{2} g^{ip} \left\{
                       g^{ks} \left(
                           \delta_s^q \delta_p^r - \delta_s^r \delta_p^q
                       \right) \nabla_k \nabla_{[r} a_{q]} +
                       g^{js} \left(
                           \delta_s^q \delta_p^r - \delta_s^r \delta_p^q
                       \right) \nabla_j \nabla_{[r} a_{q]}
                   \right\} \\
                &= g^{ip} \left(
                       g^{kq} \nabla_k \nabla_{[p} a_{q]} - g^{kr} \nabla_k \nabla_{[r} a_{p]}
                   \right) \\
                &= 2 g^{ip} g^{kq} \left(
                       \nabla_k \nabla_{[p} a_{q]}
                   \right) \\
                &= g^{ip} g^{kq} \left(
                       \nabla_k \nabla_{p} a_{q} - \nabla_k \nabla_{q} a_{p}
                   \right) \\
                &= g^{ip} \nabla_k \nabla_{p} a^k - g^{kq} \nabla_k \nabla_{q} a^i \\
                &= \left( \Grad\Div{\V{a}} - \Delta{\V{a}} \right)^i
        \end{aligned}\end{equation}
        %
        В последней строчке применено свойство перестановочности оператора абсолютного дифференцирования в евклидовом пространстве.

        Итак, мы получили общий вид оператора Лапласа в произвольной системе координат. Выпишем его отдельно:
        %
        \begin{equation}
            \Delta{\V{a}} = g^{kq} \nabla_k \nabla_{q} a^i
        \end{equation}

\end{document}
