%
%
%
%%%%%%%%%%%%%%%%%%%%%%%%%%%%%%%%%%%%%%%%%%%%%%%%%%%%%%%%%%%%%%%%%%%%%%%
%                           SECTION                                   %
%%%%%%%%%%%%%%%%%%%%%%%%%%%%%%%%%%%%%%%%%%%%%%%%%%%%%%%%%%%%%%%%%%%%%%%
%
%
%

\section{Заключение}

    В работе продемонстрировано применение методики Ли-генерации мод скалярного и векторного полей в трехмерном евклидовом пространстве. Многие математические выкладки были проделаны для риманова пространства общего вида, потому справедливы и в других классах риманова пространства, например на трехмерной сфере \cite{burlankov_tmf}.

    Применительно к векторным модам электромагнитного поля был получен вид угловых и радиальных частей сферических базовых мод и приведена методика построения угловых частей производных мод.

    Заключительным этапом стало построение кривой спектральной плотности энергии, запасенной в резонаторе. В результате чего было показано, что планковская кривая и кривая, построенная численно, накладываются друг на друга~--- первая является хорошим асимптотическим приближением последней, причем практически во всем диапазоне частот.

    Отдельно была проанализирована область $kT \sim \hbar\omega_{min}$. В ней нельзя считать спектр излучения непрерывным. Было наглядно прослежено, как ведут себя низшие моды в зависимости от температуры.

    Подводя итог, все поставленные цели в данной работе были достигнуты. На ее результатах могут базироваться более сложные работы, требующие построения мод полей б\'{о}льших размерностей в пространствах с альтернативной симметрией.
