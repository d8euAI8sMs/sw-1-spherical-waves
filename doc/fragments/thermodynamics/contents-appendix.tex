%
%
%
%%%%%%%%%%%%%%%%%%%%%%%%%%%%%%%%%%%%%%%%%%%%%%%%%%%%%%%%%%%%%%%%%%%%%%%
%                           SECTION                                   %
%%%%%%%%%%%%%%%%%%%%%%%%%%%%%%%%%%%%%%%%%%%%%%%%%%%%%%%%%%%%%%%%%%%%%%%
%
%
%

\section{Вычисление распределения мод по частотам\label{sec:getting_plank_formula}}

    В данной секции мы подробно рассмотрим вопрос получения асимптотической функции $\flatfrac{\Delta N}{\Delta \omega}$.

    Фактически данная функция является сверткой функции $N(\omega_n)$ с прямоугольным окном ширины $\Delta \omega$. Действительно, $\Delta N$ носит смысл количества мод в некоторой малой окрестности $\omega_n$, потому может быть вычислена как сумма значений функции $N(\omega_n)$ (смысл которой, напомним, количество мод, имеющих собственную частоту точно $\omega_n$) в пределах некоторого окна $\Delta \omega$. Запишем это математически:
    %
    \begin{equation}
        \Delta N(\omega) = \sum\limits_{
            \qty|\omega - \omega_n| < \Delta\omega/2
        } N(\omega_n) = \qty(W_{\Delta\omega} \vdot N)(\omega) .
    \end{equation}
    %
    Здесь $W_{\Delta\omega}$~--- прямоугольная оконная функция, $\Delta\omega$ носит характер ширины окна, операция свертки обозначена через $(f \vdot g)(x)$. Нам остается лишь только поделить $\Delta N$ на ширину окна $\Delta\omega$, чтобы получить окончательно $\flatfrac{\Delta N}{\Delta \omega}$. Следует отметить, что при таком определении $\Delta N(\omega)$ не важно, является ли аргумент $\omega$ одной из собственных частот $\omega_n$. В то время как расстояние $\omega_{n+1} - \omega_n$ не является фиксированным, мы можем получить функцию $\Delta N(\omega)$ с фиксированным шагом по аргументу или же в промежуточных точках \enquote{$\omega_{n+1/2}$}.

    До сих пор мы говорили, что последовательность $\omega_n$ нам известна. На самом деле ее получение вызывает некоторые трудности. Каждая из собственных частот резонатора $\omega^{lp}_n$ получается из граничных условий на $l$-ю моду $p$-й поляризации. Граничные условия в конечном счете требуют
    %
    \begin{equation}\begin{aligned}
        S_l(\sqrt\lambda r_\text{шара}) &= 0
            \quad \text{для $\mathrm{II}$ поляризации} , \\
        S'_l(\sqrt\lambda r_\text{шара}) &= 0
            \quad \text{для $\mathrm{I}$ поляризации},
    \end{aligned}\end{equation}
    %
    где $S_l$~--- функция Риккати-Бесселя, а $S'_l$~--- ее производная по аргументу, $\lambda = \varepsilon \flatfrac{\omega^2}{c^2}$. Если обозначить аргумент $\sqrt\lambda r_\text{шара} \sim \omega$ за $x$, то уравнения на собственные частоты резонатора перепишутся в виде
    %
    \begin{equation}\begin{aligned}
        S_l(x) &= 0, \\
        S'_l(x) &= 0.
    \end{aligned}\end{equation}
    %
    Приступим к их решению.

    Уравнения выше записаны в симметричной форме, однако учитывая то, что $S_l(x) = x j_l(x) = \sqrt{\flatfrac{\pi x}{2}} J_{l+1/2}(x)$, где $J_{l+1/2}(x)$~--- функция Бесселя первого рода полуцелого порядка $l + 1/2$, первое уравнение можно упростить до
    %
    \begin{equation}
        J_{l+1/2}(x) = 0 ,
    \end{equation}
    %
    для решения которого разработаны эффективные численные процедуры поиска корней (см. напр. \cite{handbook_of_math_functions}).

    Снова представляя $S_l(x)$ через $J_{l+1/2}(x)$, запишем эквивалентное второму уравнение
    %
    \begin{equation}
        \dv{x}(\sqrt{x} J_{l+1/2}(x)) = 0 .
    \end{equation}
    %
    Для его решения нет специальных численных схем, поэтому мы будем пытаться свести процедуру решения к известным схемам. В частности, известна процедура поиска нулей функции $J'_{l+1/2}(x)$. Уравнение выше можно рассматривать как уравнение на экстремумы функции, стоящей под знаком производной. Очевидно, что экстремумы $J_{l+1/2}(x)$ будут находиться между ее нулями. Положение экстремумов изменится не сильно, кроме того, они сместятся в одном и том же направлении, если функцию $J_{l+1/2}(x)$ умножить на монотонно возрастающую функцию $\sqrt{x}$. Это, в свою очередь, означает, что нули $J'_{l+1/2}(x)$ являются хорошим приближением для нулей $(\sqrt{x} J_{l+1/2}(x))'$. Можно построить процедуру поиска нулей целевой функции через уточнение нулей $J'_{l+1/2}(x)$.

    Можно пойти еще дальше. Несложно заметить, что корни как первого, так и второго уравнения подчиняются следующим закономерностям. Если обозначить корни первого или второго уравнения за $x^n_l$, $x^n_l < x^{n+1}_l$, то можно показать, что для одного и того же $n$ выполняется
    %
    \begin{equation}
        \frac{\pi}{4} < x^n_{l + 1} - x^n_l < \frac{\pi}{2} ,
    \end{equation}
    %
    причем приближение к левой границе осуществляется с увеличением $l$, а к правой~--- с ростом $n$. Участок $\qty[x^n_l + \flatfrac{\pi}{4}, x^n_l + \flatfrac{\pi}{2}]$ является участком монотонности целевой функции, так что на нем хорошо применимы традиционные численные методы поиска нуля, например метод Ньютона.

    Исходя из этих соображений, мы можем построить рекуррентную процедуру численного поиска корней обоих уравнений. Зададимся некоторыми значениями $l_{min}$, $l_{max}$ и $n_{max}$. Тогда для первого уравнения процедура будет сводиться к следующему:
    %
    \begin{enumerate}[nosep]
        \item ищутся корни $x^n_{l_{min}}$, $n = 1, 2, \dots, n_{max}$ функции $J_{l+1/2}(x)$ известными специальными методами;
        \item поиск корней $x^n_{l+1}$, $n = 1, 2, \dots, n_{max}$ организуется в интервале $\qty[x^n_l + \flatfrac{\pi}{4}, x^n_l + \flatfrac{\pi}{2}]$ обычными численными методами, пока не будут найдены корни для всех $l \le l_{max}$.
    \end{enumerate}
    %
    Для второго уравнения добавится лишь один дополнительный шаг:
    %
    \begin{enumerate}[nosep]
        \item за начальное приближение $x^n_{l_{min}}$, $n = 1, 2, \dots, n_{max}$ выбираются нули $J'_{l+1/2}(x)$,
        \item они уточняются для целевой функции в интервале $\qty[x^n_{l_{min}} + \flatfrac{\pi}{4}, x^n_{l_{min}} + \flatfrac{\pi}{2}]$ обычными численными методами,
        \item поиск корней $x^{l+1}_n$, $n = 1, 2, \dots, n_{max}$ организуется в интервале $\qty[x^n_l + \flatfrac{\pi}{4}, x^n_l + \flatfrac{\pi}{2}]$ как в предыдущем случае.
    \end{enumerate}

    Поговорим о сам\'{о}й функции $N(\omega_n)$. Мы знаем, что $l$-я мода $2l + 1$-кратно вырождена по энергии. Это равносильно утверждению, что одной собственной энергии $\omega^{lp}_n$ соответствует $2l + 1$ мода. Поэтому значение функции $N(\omega^{lp}_n) = 2l + 1$. Функция $N(\omega_n)$ изображена на \autoref{fig:n}.

    Ранее отмечалось, что в области $0 < \omega < \Omega$, где $\Omega = \min_p{\omega^{l_{max}p}_{i_{min}}}$~--- первый нуль $l_{max}$-й радиальной функции одной из поляризаций, мы имеем полную информацию о распределении мод по частотам. Действительно, в области $\omega > \Omega$ отсутствуют нули функций б\'{о}льших порядков, поэтому она не содержит полной информации и непригодна для дальнейшего исследования. Подробно рассмотрим первую область.

    Найдем функцию $(\flatfrac{\Delta N}{\Delta \omega})(\omega)$ описанным выше способом, пока выбирая в качестве $\omega$ значения собственных частот резонатора $\omega_n$. Мы обнаружим, что в области $\Delta\omega / 2 < \omega < \Omega - \Delta\omega / 2$ (поправка на полуширину окна вводится для устранения \enquote{краевых эффектов} свертки) данная функция хорошо аппроксимируется квадратичной функцией (\autoref{fig:dndx}), причем качество аппроксимации увеличивается с увеличением доступного $l_{max}$: коэффициент корреляции Пирсона стремится к единице\footnotemark{}. Это говорит об асимптотическом характере зависимости $(\Delta N)(\omega) \sim \omega^2$. В области вблизи нуля (\autoref{fig:dndx}\subref{fig:dndx_mag}) все же не наблюдается значительных отклонений $\flatfrac{\Delta N}{\Delta \omega}$ от аппроксимирующей квадратичной функции, потому можно говорить, что с большой точностью $(\Delta N)(\omega) \sim \omega^2$ верно и для близких к нулю собственных частот.

    \footnotetext{
        Конкретные цифры в работе не приводятся. Они зависят от выбранной методики проведения расчетов, их точности, от вида функционала ошибки в аппроксимационной схеме и т.д. На наш взгляд достаточно и того, что визуально зависимости накладываются друг на друга, причем коэффициент при $\omega^2$ с некоторой точностью совпадает с теоретическим. Отметим лишь, что уже при $l_\text{max} = 50$ и ширине окна $\Delta \omega \sim \Delta x = 10$ ошибка в определении коэффициента не превышала $0.6\%$ при коэффициенте корреляции около $0.9998$. При увеличении $l_\text{max}$ точность соответствия только возрастала.
    }

    Поскольку $\flatfrac{\Delta N}{\Delta \omega}$ можно с высокой точностью описывать квадратичным законом в широких пределах изменения $\omega$, спектральная плотность энергии будет очень хорошо ложиться на планковскую кривую даже в области близких к нулю частот. Запишем выражение для $u(\omega)$ в единицах переменной $x$:
    %
    \begin{equation}\begin{aligned}
        u(x)
            &= \frac{1}{\pi^2 r_\text{шара}^3}
                x n(x) \qty(\frac{\Delta N}{\Delta x})(x) \\
            &= \frac{\flatfrac{x^3}{\pi^2 r_\text{шара}^3}}{
                    \exp\qty(\frac{\hbar v x}{kT r_\text{шара}}) - 1
                } .
    \end{aligned}\end{equation}
    %
    Кривая $u(x)$ для некоторого значения $r_\text{шара}$ изображена на \autoref{fig:plank}. Необходимо помнить о том, что $u(\omega)$~--- спектральная плотность энергии \textit{на единицу объема}, в то время как найденная выше функция $\flatfrac{\Delta N}{\Delta x}$ приходится \textit{на весь объем резонатора}. Поэтому перед подстановкой в данную формулу $\flatfrac{\Delta N}{\Delta x}$ следует поделить на объем резонатора, т.е. на величину $V_\text{шара} = \flatfrac{4}{3} \pi r_\text{шара}^3$.
