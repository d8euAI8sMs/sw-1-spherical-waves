%
%
%
%%%%%%%%%%%%%%%%%%%%%%%%%%%%%%%%%%%%%%%%%%%%%%%%%%%%%%%%%%%%%%%%%%%%%%%
%                           SUBSECTION                                %
%%%%%%%%%%%%%%%%%%%%%%%%%%%%%%%%%%%%%%%%%%%%%%%%%%%%%%%%%%%%%%%%%%%%%%%
%
%
%

\subsection*{Введение}

    В данной части работы будет изложено введение в сферические резонаторы и их термодинамику.

    Изучение термодинамики подобных структур важно для современной микроэлектроники, оптики и других наук, оперирующих со сферически симметричными малоразмерными структурами. В частности, оно позволяет получить более или менее точные количественные оценки мешающих факторов~--- тепловых шумов, а также указать возможные способы компенсации их воздействия.

    Общим предметом рассмотрения работы станут сферические резонаторы, ограниченные идеально проводящими стенками. Будет получен вид векторных сферических мод, описана методика их получения. Исходя из этого будет получена спектральная плотность энергии, запасенной в резонаторе.
