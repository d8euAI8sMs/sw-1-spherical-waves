%
%
%
%%%%%%%%%%%%%%%%%%%%%%%%%%%%%%%%%%%%%%%%%%%%%%%%%%%%%%%%%%%%%%%%%%%%%%%
%                           SUBSECTION                                %
%%%%%%%%%%%%%%%%%%%%%%%%%%%%%%%%%%%%%%%%%%%%%%%%%%%%%%%%%%%%%%%%%%%%%%%
%
%
%

\subsection*{Введение}

    Целью данной работы является построение мод электромагнитного поля в сферическом резонаторе.

    Классические методики решения данной задачи в значительной степени эвристичны, другие приводят к весьма громоздким результатам. К тому же их применение к полям б\'{о}льшей тензорной размерности весьма проблематично. \cite{burlankov_tmf}

    В данной работе приводится новый подход, в основе которого лежат векторы Киллинга пространства, в котором распространяется поле. Данный подход учитывает естественную симметрию объемлющего пространства, тем самым требует более коротких выкладок. К тому же он естественным образом обобщается для описания полей любой тензорной размерности, например гравитационного. \cite{burlankov_tmf}

    Впервые указанная методика была применена для описания полей различной тензорной размерности на трехмерной сфере \cite{burlankov_tmf}.
