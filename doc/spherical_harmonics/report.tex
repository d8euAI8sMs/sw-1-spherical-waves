\documentclass[12pt,a4paper]{article}

\usepackage[T1,T2A]{fontenc}
\usepackage[utf8]{inputenc}
\usepackage[english,russian]{babel}
\usepackage{microtype}
\usepackage{csquotes}
\usepackage{amsmath}
\usepackage{amsthm}
\usepackage{amssymb}
\usepackage{mathtext}
\usepackage{physics}
\usepackage{newfloat}
\usepackage{caption}
\usepackage{indentfirst}
\usepackage{titlesec,titletoc}
\usepackage{geometry}
\usepackage{hyperref}
\usepackage{mdframed}
\usepackage[inline]{enumitem}
\usepackage{graphicx}
\usepackage{subfig}
\usepackage[titletoc,toc]{appendix}

\DeclareGraphicsExtensions{.pdf,.png,.jpg,.PNG}
\graphicspath{{./img/}}
\captionsetup[figure]{justification=centering}
\renewcommand{\thesubfigure}{\asbuk{subfigure}}
\DeclareCaptionLabelSeparator{dotseparator}{. }
\captionsetup{labelsep=dotseparator}
\geometry{left=1cm,right=1cm,top=2cm,bottom=2cm}
\makeatletter\appto{\appendices}{\def\Hy@chapapp{Appendix}}\makeatother
\renewcommand{\appendixtocname}{Приложения}
\renewcommand{\appendixpagename}{Приложения}

\DeclareMathOperator{\Rot}{\mathbf{rot}}
\DeclareMathOperator{\Grad}{\mathbf{grad}}
\DeclareMathOperator{\Div}{\mathbf{div}}
\DeclareMathOperator{\D}{D}
\newcommand{\V}[1]{\mathbf{#1}}
\newcommand{\Op}[1]{\hat{\V{#1}}}


\title{Сферические гармоники}
\author{Василевский А.В.}

\begin{document}

    \maketitle
    \tableofcontents

    %
    %
    %
    %%%%%%%%%%%%%%%%%%%%%%%%%%%%%%%%%%%%%%%%%%%%%%%%%%%%%%%%%%%%%%%%%%%%%%%
    %                           SECTION                                   %
    %%%%%%%%%%%%%%%%%%%%%%%%%%%%%%%%%%%%%%%%%%%%%%%%%%%%%%%%%%%%%%%%%%%%%%%
    %
    %
    %

    \section*{Введение}

        Целью данной работы является построение мод электромагнитного поля в сферическом резонаторе.

        Классические методики решения данной задачи в значительной степени эвристичны, другие приводят к весьма громоздким результатам. К тому же их применение к полям б\'{о}льшей тензорной размерности весьма проблематично.

        В данной работе приводится новый подход, в основе которого лежат векторы Киллинга пространства, в котором распространяется поле. Данный подход учитывает естественную симметрию объемлющего пространства, тем самым требует более коротких выкладок. К тому же он естественным образом обобщается для описания полей любой тензорной размерности, например гравитационного.

        Впервые указанная методика была применена для описания полей различной тензорной размерности на трехмерной сфере \cite{burlankov_tmf}.

    %
    %
    %
    %%%%%%%%%%%%%%%%%%%%%%%%%%%%%%%%%%%%%%%%%%%%%%%%%%%%%%%%%%%%%%%%%%%%%%%
    %                           SECTION                                   %
    %%%%%%%%%%%%%%%%%%%%%%%%%%%%%%%%%%%%%%%%%%%%%%%%%%%%%%%%%%%%%%%%%%%%%%%
    %
    %
    %

    \section{Уравнение на сферические моды}

        Уравнение на сферические моды получается из волнового уравнения, которое, в свою очередь, получается из стационарных уравнений Максвелла в среде.

        В изотропной немагнитной среде, где справедливо $\V{D} = \varepsilon \V{E}$, волновое уравнение для вектора $\V{E}$ принимает вид:
        %
        \begin{equation}\label{eq:wave_equation}
            \Delta \V{E} = - \varepsilon \frac{\omega^2}{c^2} \V{E} .
        \end{equation}
        %
        Оно является уравнением на собственные функции и собственные значения (моды) оператора Лапласа $\Delta$, что позволяет эффективно применить аппарат теории операторов для его решения.

        В задаче о сферическом резонаторе спектр мод дискретен и определяется двумя числами, $l$ и $m$, т.е. общее решение уравнения \autoref{eq:wave_equation} выражается через линейную комбинацию полученных мод: $\V{E} = \sum a_{l,m} \V{E}_{l,m}$.

    %
    %
    %
    %%%%%%%%%%%%%%%%%%%%%%%%%%%%%%%%%%%%%%%%%%%%%%%%%%%%%%%%%%%%%%%%%%%%%%%
    %                        BIBLIOGRAPHY                                 %
    %%%%%%%%%%%%%%%%%%%%%%%%%%%%%%%%%%%%%%%%%%%%%%%%%%%%%%%%%%%%%%%%%%%%%%%
    %
    %
    %

    \nocite{*}
    \bibliographystyle{../../lib/doc/bib/utf8gost705s}
    \bibliography{../../lib/doc/bib/physics,math}

\end{document}
